\documentclass[pdf,utf8,russian,aspectratio=169]{beamer}
\usepackage[T2A]{fontenc}
\usetheme{Copenhagen}
\usepackage{setspace}
\usepackage{amsmath}
\usepackage{pgfplots}
\usepackage[utf8]{inputenc}
\usepackage{tikz-cd}
\usepackage[all, 2cell]{xy}
\usepackage{amssymb}
\usepackage{verbatim}
\usepackage[all]{xy}
\usepackage{tikz}
\usepackage{bussproofs}
\usepackage{dsfont}
\usepackage{mathabx}
\usepackage{animate}
\usetikzlibrary{graphs}
\usetikzlibrary{arrows}
\usepackage{hyperref}
\usepackage[english,russian]{babel}
\usepackage{listings}
\usepackage{color}
\usepackage[all, 2cell]{xy}
\usepackage[all]{xy}
\usepackage{listings}
\usepackage{mathrsfs}
\newtheorem{defin}{Определение}
\newtheorem{theor}{Теорема}
\newtheorem{lem}{Лемма}
\title{Введение в модальную логику, \\ Лекция 1}
\author{Даня Рогозин \\ МГУ, Serokell}
\date{19 октября \\ Computer Science Club \\ ПОМИ РАН}
\begin{document}

\maketitle

\begin{frame}
  \frametitle{Эпиграф}

  \begin{center}
    Необходимость непреклонна

\vspace{\baselineskip}

    Эврипид, 480-е --- 406 до н. э., \\
    финальные слова из трагедии "Гекуба"
  \end{center}
  \begin{center}
  \includegraphics[scale=0.25]{euripides.jpg}
  \end{center}
\end{frame}

\begin{frame}
  \frametitle{Что такое модальная логика?}

  В рамках курса мы дадим точное математическое определение модальной логики. Пока что мы определим ее неформально.
  Неформально, модальная логика существует в двух смыслах:

  \begin{itemize}
    \onslide<2->{\item \emph{В широком смысле}, модальная логика является ветвью неклассической логики, которая изучает
    утверждения, дополненные специальными качествами, как доказуемость, необходимость, возможность, и так далее}
    \onslide<3->{\item \emph{В узком смысле}, под модальной логикой подразумевают конкретную логику, которая расширяет (как правило) классическую логику дополнительными модальными операторами. То есть, в узком смысле, модальная логика --- это некоторое отдельно взятое исчисление}
  \end{itemize}
\end{frame}

\begin{frame}
  \frametitle{Очень краткая история}
  \begin{itemize}
  \onslide<1->{\item Системы с модальностями впервые стал описывать Кларенс Льюис в начале 20-го века, который ввел в оборот связки $\Box$ и $\Diamond$, обозначающие необходимость и возможность соответственно. Тогда у этих модальностей не было семантики, связки понимались интуитивно}
  \onslide<2->{\item В 1940-е годы некоторые из систем модальной логики изучались для описания топологических и метрических пространств. Это направление берет начало в работах Тарского и Маккинси}
  \onslide<3->{\item В конце 1950-х годов Сол Крипке предложил строить реляционные модели для модальной и интуиционистской логик}
  \onslide<4->{\item 1960-е--1970-е (что верно и по сей день) модальная логика эмансипируется от своих философских корней и становится математической самостоятельной дисциплиной, что сложилось в результате исследований Сегерберга, Салквиста, Гольдблатта, Томасона, Габбая, и т. д.}
  \end{itemize}
\end{frame}

\begin{frame}
  \frametitle{Место модальной логики в общем контексте}

  В математической логике можно выделить три основных раздела, в контексте которых существует и модальная логика:

  \begin{itemize}
    \onslide<2->{\item \emph{Теория доказательств} изучает свойства и структуру логических выводов, доказуемость в тех или иных формальных
    теориях}
    \onslide<3->{\item \emph{Теория моделей} изучает семантику логических формул и их математическую интерпретацию}
    \onslide<4->{\item \emph{Теория алгоритмов} ставит вопрос о вычислимости за конечное число шагов и о затратах ресурсов на выполнение той или иной процедуры}
  \end{itemize}
\end{frame}

\begin{frame}
  \frametitle{Модальный язык}

  \begin{defin}
    \onslide<1->{Пусть $\mathbb{V} = \{ p, q, r, ... \}$ счетное множество пропозициональных переменных. Множество формул $Fm$ определяется
    индуктивно:}

    \begin{enumerate}
      \onslide<2->{\item Всякая переменная --- это формула
      \item Если $\phi$ --- это формула, то $\neg \phi$ --- это формула
      \item Если $\phi, \psi$ --- это формулы, то $(\phi \to \psi)$, $(\phi \land \psi)$, $(\phi \lor \psi)$ --- это формулы
      \item Если $\phi$ --- это формула, то $\Box \phi$ --- это формула}
    \end{enumerate}

    \onslide<3->{Также у нас будет связка $\Diamond$, которую мы введем как сокращение $\Diamond \phi = \neg \Box \neg \phi$.}
  \end{defin}
\end{frame}

\begin{frame}
  \frametitle{Модальный язык}
  Связки $\neg$, $\to$, $\land$ и $\lor$ мы понимаем классически. Остается вопрос: как понимать $\Box$?

  Существует масса способов, вот некоторые из них:

  \begin{enumerate}
    \item $\Box \phi$ --- "$\phi$ необходимо". (алетическая модальность)
    \item $\Box \phi$ --- $\Box$ как предикат доказуемости в арифметической теории $\bf T$ (например, в арифметике Пеано)
    \item $\Box \phi$ --- $\Box$ как внутренность множества в топологическом пространстве (топологическая модальность, моя любимая)
    \item $\Box \phi$ --- "всегда будет $\phi$" (временная модальность)
  \end{enumerate}
\end{frame}

\begin{frame}
  \frametitle{Шкала Крипке}

  \begin{defin}
    Шкала Крипке --- это пара $\mathcal{F} = \langle W, R \rangle$, где $W$ непустое множество, а $R \subseteq W \times W$ --- это бинарное отношение на множестве $W$
  \end{defin}

\begin{itemize}
  \onslide<1->{\item Множество $W$ называют \emph{носителем шкалы}. Более философски, $W$ --- это множество возможных миров или положений дел.}
  \onslide<2->{\item Отношение $R$ часто называют \emph{отношением достижимости}. Если $x$ находится $y$ в отношении $R$, то мы будем говорить, что "$x$ видит $y$" и писать $x R y$}
  \onslide<3->{\item Можно также считать шкалу Крипке ориентированным графом, в котором $R$ --- это множество ребер в данном графе.}
\end{itemize}

\onslide<4->{Также мы будет использовать обозначение $R(x) = \{ y \: | \: x R y \}$}
\end{frame}

\begin{frame}
  \frametitle{Модель Крипке}
\begin{defin}
  Пусть $\mathcal{F} = \langle W, R \rangle$ --- это шкала Крипке. Модель Крипке --- это упорядоченная пара $\mathcal{M} = \langle \mathcal{F}, \vartheta \rangle$, где $\vartheta : \mathbb{V} \to 2^{W}$ --- это функция оценки.

  \onslide<2->{Отношение истинности (также, отношение вынуждения) $\mathcal{M}, w \models \phi$ определяется индукцией по формуле $\phi$:}

  \begin{itemize}
    \onslide<3->{
    \item $\mathcal{M}, w \models p \Leftrightarrow w \in \vartheta(p)$
    \item $\mathcal{M}, w \models \neg \phi \Leftrightarrow \mathcal{M}, w \nvDash \phi$
    \item $\mathcal{M}, w \models \phi \lor \psi \Leftrightarrow \mathcal{M}, w \models \phi \text{ или } \mathcal{M}, w \models \psi$
    \item $\mathcal{M}, w \models \phi \land \psi \Leftrightarrow \mathcal{M}, w \models \phi \text{ и } \mathcal{M}, w \models \psi$
    \item $\mathcal{M}, w \models \Box \phi \Leftrightarrow \forall v \:\: w R v \Rightarrow \mathcal{M}, v \models \phi$}
  \end{itemize}
\end{defin}
\end{frame}

\begin{frame}
  \frametitle{Модель Крипке}

  \onslide<1->{Неформально прокомментируем условие:
  \begin{itemize}
  \item $\mathcal{M}, w \models \Box \phi \Leftrightarrow \forall v \:\: w R v \Rightarrow \mathcal{M}, v \models \phi$
\end{itemize}}

  \onslide<2->{Здесь бокс удобнее всего читать через необходимость.}

  \onslide<3->{$\phi$ истинно с необходимостью при положении дел $w$, если при любом развитии событий $v$, к которому можно прийти из $w$, утверждение $\phi$ будет истинно.}

\onslide<4->{
  \xymatrix{
  &&&& v_1 \models \phi \\
  &&& v_4 \models \phi & w \ar[u]^{R} \ar[d]_{R} \ar[l]_{R} \ar[r]^{R} \models \Box \phi & v_2 \models \phi \\
  &&&& v_3 \models \phi
  }
  }
\end{frame}

\begin{frame}
  \frametitle{Пример модели Крипке}
\onslide<1->{
\xymatrix{
&&&&&&& b \ar[dd]^{R} \ar@(ul,ur)^{R} \\
&&&&& a \ar[urr]^{R} \ar[drr]_{R} \\
&&&&&&& c \ar@(dr,dl)^{R} \\
}}

\onslide<2->{
Пусть $\vartheta(p) = \{ a, b, c \}$ и $\vartheta(q) = \{ b \}$.
}

\onslide<3->{
Тогда
}
\begin{enumerate}
\onslide<4->{
\item $a \models \Box p, \Box p \to \Box \Box p, \Diamond p \to \Diamond \Box p$, etc
\item $b \models \Box p \to p, \Box q \to q$
\item $c \models \Box p \to p$
\item $\mathcal{M} \models \Box p \to \Diamond p$}
\end{enumerate}
\end{frame}

\begin{frame}
  \frametitle{Истинность в модели и общезначимость в шкале}
\onslide<1->{
  \begin{defin}
    Формула $\phi$ \emph{истинна в модели} $\mathcal{M}$, если для любой точки $w \in W$, $\mathcal{M}, w \models \phi$.
    Кратко, $\mathcal{M} \models \phi$.
  \end{defin}
}

\onslide<2->{
  \begin{defin}
    Формула $\phi$ \emph{общезначима в шкале} $\mathcal{F}$, если для любой оценки $\vartheta$, в любой модели $\mathcal{M} = \langle \mathcal{F}, \vartheta \rangle$ формула $\phi$ истинна, то есть $\mathcal{M} \models \phi$.

    Обозначение $\mathcal{F} \models \phi$
  \end{defin}
}
\end{frame}

\begin{frame}
  \frametitle{Логика класса шкал}

\onslide<1->{
  \begin{defin}
    Пусть $\mathcal{F}$ --- это шкала Крипке. \emph{Логикой} шкалы $\mathcal{F}$ называется множество формул, общезначимых в данной шкале. Обозначение $\operatorname{Log}(\mathcal{F}) = \{ \phi \in Fm \: | \: \mathcal{F} \models \phi \}$
  \end{defin}
}

\onslide<2->{
  \begin{defin}
    Пусть $\mathbb{F}$ --- это класс шкал. \emph{Логикой класса шкал} $\mathbb{F}$ называется множество формул, общезначимых в каждой из шкал в данном классе. Обозначение $\operatorname{Log}(\mathbb{F}) = \bigcap \limits_{\mathcal{F} \in \mathbb{F}} \operatorname{Log}(\mathcal{F})$.
  \end{defin}
}
\end{frame}

\begin{frame}
  \frametitle{Модальные формулы и свойства шкал}

  Можно исследовать взаимосвязи общезначимости тех или иных формул со свойствами шкал, на которых данные формулы общезначимы. Введем следующие формулы:

  \begin{itemize}
    \item {\bf AT} $\Box p \to p$ (рефлексивность)
    \item {\bf A4} $\Box p \to \Box \: \Box p$ (транзитивность)
    \item {\bf AB} $p \to \Box \Diamond p$ (симметричность)
    \item {\bf ACR} $\Diamond \Box p \to \Box \Diamond p$ (формула Черча-Россера)
    \item {\bf AD} $\Diamond \top$ (сериальность)
  \end{itemize}
\end{frame}

\begin{frame}
    \frametitle{Модальные формулы и свойства шкал}

\onslide<1->{
    \begin{lem} Пусть $\mathcal{F} = \langle W, R \rangle$ --- это шкала Крипке, тогда}
      \begin{enumerate}
        \onslide<2->{\item $\mathcal{F} \models \Box p \to p \Leftrightarrow \forall x \in W \:\: x R x$}
        \onslide<3->{\item $\mathcal{F} \models \Box p \to \Box \: \Box p \Leftrightarrow \forall x, y, z \in W \:\: x R y \: \& \: y R z \Rightarrow x R z$}
        \onslide<4->{\item $\mathcal{F} \models p \to \Box \Diamond p \Leftrightarrow \forall x, y \in W \:\: x R y \Rightarrow y R x$}
        \onslide<5->{\item $\mathcal{F} \models \Diamond \Box p \to \Box \Diamond p \Leftrightarrow
        \forall x, y, z \in W \:\: x R y \: \& \: x R z \Rightarrow \exists z_1 \in W \: y R z_1 \: \& \: z R z_1$}
        \onslide<6->{\item $\mathcal{F} \models \Diamond \top \Leftrightarrow \forall x \in W \: \exists y \in W \:\: x R y$}
      \end{enumerate}
    \end{lem}

\onslide<7->{
    \begin{proof}
      Рассмотрим эквивалентности (1), (2) и (5).
    \end{proof}
}
\end{frame}

\begin{frame}
    \frametitle{Модальные формулы и свойства шкал}
    \begin{lem} Пусть $\mathcal{F} = \langle W, R \rangle$ --- это шкала Крипке, тогда

\begin{center}
      $\mathcal{F} \models \Box p \to p \Leftrightarrow \forall x \in W \:\: x R x$
\end{center}
    \end{lem}

Для удобства перепишем эту формулу в эквивалентный вид $p \to \Diamond p$.

\onslide<1->{
    \begin{proof}
      $(\Rightarrow)$ Пусть $\mathcal{F} \models p \to \Diamond p$. Рассмотрим оценку $\vartheta(p) = \{ x \}$, где $x \in W$.}
      \onslide<2->{Тогда $\mathcal{M}, x \models p$ и $\mathcal{M}, x \models \Diamond p$. Тогда найдется $y \in R(x)$, такой, что $\mathcal{M}, y \models p$.}
      \onslide<3->{Но $p$ истинно только в $x$, тогда $x R x$.
    \end{proof}
    }
\end{frame}

\begin{frame}
    \frametitle{Модальные формулы и свойства шкал}
    \begin{lem} Пусть $\mathcal{F} = \langle W, R \rangle$ --- это шкала Крипке, тогда

\begin{center}
      $\mathcal{F} \models \Box p \to p \Leftrightarrow \forall x \in W \:\: x R x$
\end{center}
    \end{lem}

    \begin{proof}
      \onslide<1->{$(\Leftarrow)$ Пусть $\mathcal{F} = \langle W, R \rangle$ рефлексивная шкала и $\mathcal{F} \nvDash p \to \Diamond p$.}
      \onslide<2->{Тогда найдется такая оценка $\vartheta$, модель $\mathcal{M} = \langle \mathcal{F}, \vartheta \rangle$ и $x \in W$, что
      $\mathcal{M}, x \models p$ и $\mathcal{M}, x \nvDash \Diamond p$.} \onslide<3->{Из $\mathcal{M}, x \nvDash \Diamond p$ следует, что для любого
      $y \in R(x)$, $\mathcal{M}, y \nvDash p$.} \onslide<4->{Так как $\mathcal{F}$ рефлексивная шкала, тогда $x R x$ и $x \in R(x)$, тогда $\mathcal{M}, x \nvDash p$. Противоречие.}
    \end{proof}
\end{frame}

\begin{frame}
  \frametitle{Модальные формулы и свойства шкал}

  \begin{lem} Пусть $\mathcal{F} = \langle W, R \rangle$ --- это шкала Крипке, тогда

\begin{center}
    $\mathcal{F} \models \Box p \to \Box \: \Box p \Leftrightarrow \forall x, y, z \in W \:\: x R y \: \& \: y R z \Rightarrow x R z$
\end{center}
  \end{lem}

  \onslide<1->{Аналогично, перепишем нашу формулу как $\Diamond \Diamond p \to \Diamond p$.}

\begin{proof}
  \onslide<2->{($\Rightarrow$)} \onslide<3->{Пусть $\mathcal{F} \models \Diamond \Diamond p \to \Diamond p$ и $x, y, z \in W$, такие, что $x R y$ и $y R z$.} \onslide<3->{Рассмотрим оценку
  $\vartheta(p) = \{ z \}$.} \onslide<4->{Так как $x R y R z$, тогда $\mathcal{M}, x \models \Diamond \Diamond p$.} \onslide<5->{Тогда $\mathcal{M}, x \models \Diamond p$.}
  \onslide<6->{Значит, найдется $x' \in R(x)$, что $\mathcal{M}, x' \models p$. Но $p$ истинно только $z$, тогда $x R z$}
\end{proof}
\end{frame}

\begin{frame}
  \frametitle{Модальные формулы и свойства шкал}

  \begin{lem} Пусть $\mathcal{F} = \langle W, R \rangle$ --- это шкала Крипке, тогда

\begin{center}
    $\mathcal{F} \models \Box p \to \Box \: \Box p \Leftrightarrow \forall x, y, z \in W \:\: x R y \: \& \: y R z \Rightarrow x R z$
\end{center}
  \end{lem}

\begin{proof}
  \onslide<1->{($\Leftarrow$) Пусть $\mathcal{F} = \langle W, R \rangle$ --- это транзитивная шкала,
  $\vartheta$ --- это оценка, модель $\mathcal{M} = \langle \mathcal{F}, \vartheta \rangle$ и $x \in W$.}
  \onslide<2->{Предположим, что $\mathcal{M}, x \models \Diamond \Diamond p$. Тогда найдется $y \in R(x)$, что $\mathcal{M}, y \models \Diamond p$. Тогда найдется
  найдется $z \in R(x)$, что $\mathcal{M}, z \models p$. $x R y$ и $y R z$, тогда $x R z$.}
  \onslide<3->{Значит, $\mathcal{M}, x \models \Diamond p$.}
\end{proof}
\end{frame}

\begin{frame}
  \frametitle{Модальные формулы и свойства шкал}

  \begin{lem} Пусть $\mathcal{F} = \langle W, R \rangle$ --- это шкала Крипке, тогда

\begin{center}
    $\mathcal{F} \models \Diamond \top \Leftrightarrow \forall x \in W \: \exists y \in W \:\: x R y$
\end{center}
  \end{lem}

  \begin{proof}
    $(\Rightarrow)$ Пусть $\mathcal{F} \models \Diamond \top$. Тогда $\mathcal{M}, x \models \Diamond \top$. Тогда найдется $y \in R(x)$, что $\mathcal{M}, y \models \top$. Так как $x$ произвольный, то для каждого такого $x$ найдется последователь.
  \end{proof}
\end{frame}

\begin{frame}
\frametitle{Модальные формулы и свойства шкал}

\begin{lem} Пусть $\mathcal{F} = \langle W, R \rangle$ --- это шкала Крипке, тогда

  \begin{center}
      $\mathcal{F} \models \Diamond \top \Leftrightarrow \forall x \in W \: \exists y \in W \:\: x R y$
  \end{center}
\end{lem}

\onslide<1->{
\begin{proof}
  $(\Leftarrow)$} \onslide<2->{Пусть $\mathcal{F}$ сериальна и при этом $\mathcal{M} \nvDash \Diamond \top$.
  Тогда при некотором $x$, $\mathcal{M}, x \nvDash \Diamond \top$. То есть, $\mathcal{M}, x \models \neg \Diamond \top$. По эквивалентности, $\mathcal{M}, x \models \neg \neg \Box \neg \top$.} \onslide<3->{Иными словами,
  $\mathcal{M}, x \models \Box \bot$.} \onslide<4->{Откуда, $R(x) = \emptyset$. Но шкала сериальна. Противоречие.}
\end{proof}

\end{frame}

\begin{frame}
  \frametitle{Модально неопределимые свойства шкал}

\begin{itemize}
  \onslide<1->{\item Заметим, что не всякое свойство шкал является модально определимым.}
  \onslide<2->{\item Покажем, что не существует модальной формулы $\phi$, такой, что в шкале $\mathcal{F} \models \phi$ тогда и только тогда, когда отношение $R$ иррефлексивно}
  \onslide<3->{\item Для это введем понятие $p$-морфизма, естественного гомоморфизма шкал и моделей Крипке}
\end{itemize}
\end{frame}

\begin{frame}
  \frametitle{$p$-морфизм шкал}
  \begin{defin} $p$-морфизмом шкал Крипке называется отображение $f : \mathcal{F}_1 \to \mathcal{F}_2$, такое что:
    \begin{enumerate}
      \onslide<1->{
      \item $f$ --- монотонно, $a R_1 b \Rightarrow f (a) R_2 f (b)$, где $a, b \in W_1$
      \item $f$ обладает свойством поднятия, то есть, $f(a) R_2 c \Rightarrow \exists b \in W_1 \: a R_1 b \: \& \: f(b) = c$:

      \begin{small}
      \xymatrix{
      &&&& a \ar@{|->}[d] \ar@{-->}[rr]^{R_1} && \exists b \ar@{|-->}[d] \\
      &&&& f(a) \ar[rr]_{R_2} && c
      }}
    \end{small}
  \end{enumerate}

  \onslide<2->{Также нас будут интересовать сюръективные $p$-морфизмы. То есть такие $f : \mathcal{F}_1 \to \mathcal{F}_2$, что $\forall y \in \mathcal{F}_2 \: \exists x \in \mathcal{F}_1 \:\: f(x) = y$. Обозначение $f : \mathcal{F}_1 \twoheadrightarrow \mathcal{F}_2$.}
  \end{defin}
\end{frame}

\begin{frame}
  \frametitle{$p$-морфизм моделей}
  \begin{defin}
    \onslide<1->{Пусть $\mathcal{M}_1 = \langle \mathcal{F}_1, \vartheta_1 \rangle$ и $\mathcal{M}_2 = \langle \mathcal{F}_2,
    \vartheta_2 \rangle$ --- это модели Крипке.} \onslide<2->{$p$-морфизмом моделей Крипке $f : \mathcal{M}_1 \to \mathcal{M}_2$ называется $p$-морфизм шкал $f : \mathcal{F}_1 \to \mathcal{F}_2$ со следующим условием:}

\onslide<3->{
    \begin{center}
      $\mathcal{M}_1, w \models p \Leftrightarrow \mathcal{M}_1, f(w) \models p$, для всех $w \in \mathcal{M}_2$, где $p$ --- это переменная.
    \end{center}
}
  \end{defin}

\onslide<4->{
  $p$-морфизмы позволяют отображать модели и шкалы, сохраняя истинность формул. Сформулируем лемму:
  }
\end{frame}

\begin{frame}
  \frametitle{Лемма о $p$-морфизмах}

  \begin{lem}
    \begin{enumerate}
      \item Пусть $\mathcal{M}_1, \mathcal{M}_2$ --- модели Крипке и $f : \mathcal{M}_1 \to \mathcal{M}_2$. Тогда
      $\mathcal{M}_1, w \models \phi \Leftrightarrow \mathcal{M}_2, f(w) \models \phi$
      \item Пусть $f: \mathcal{F}_1 \twoheadrightarrow \mathcal{F}_2$, тогда $\operatorname{Log}(\mathcal{F}_1) \subseteq \operatorname{Log}(\mathcal{F}_2)$
    \end{enumerate}
  \end{lem}
\end{frame}

\begin{frame}
  \frametitle{Лемма о $p$-морфизмах}

  \onslide<1->{Докажем первый пункт

  \begin{lem}
    Пусть $\mathcal{M}_1, \mathcal{M}_2$ --- модели Крипке и $f : \mathcal{M}_1 \to \mathcal{M}_2$. Тогда
    $\mathcal{M}_1, w \models \phi \Leftrightarrow \mathcal{M}_2, f(w) \models \phi$
  \end{lem}
}

\onslide<2->{
Индукция по построению формулы $\phi$. Рассмотрим случай, когда $\phi \eqcirc \Diamond \psi$. Покажем, что $\mathcal{M}_1, w \models \Diamond \psi \Leftrightarrow \mathcal{M}_2, f(w) \models \Diamond \psi$
}

\onslide<3->{
  \begin{proof}
    $(\Rightarrow)$ Пусть $\mathcal{M}_1, w \models \Diamond \psi$. Тогда найдется $u \in R_1 (w)$ такой, что $\mathcal{M}_1, u \models \psi$. По предположению индукциию, $\mathcal{M}_2, f(u) \models \psi$. По монотонности, $w R u \Rightarrow f(w) R_2 f(u)$. Тогда $\mathcal{M}_2, f(w) \models \Diamond \psi$.
  \end{proof}
}
\end{frame}

\begin{frame}
  \frametitle{Лемма о $p$-морфизмах}

  Докажем первый пункт

  \begin{lem}
    Пусть $\mathcal{M}_1, \mathcal{M}_2$ --- модели Крипке и $f : \mathcal{M}_1 \to \mathcal{M}_2$. Тогда
    $\mathcal{M}_1, w \models \phi \Leftrightarrow \mathcal{M}_1, f(w) \models \phi$
  \end{lem}

\onslide<1->{
  \begin{proof}
    $(\Leftarrow)$} \onslide<2->{Пусть $\mathcal{M}_2, f(w) \models \Diamond \psi$. Тогда $w' \in R_2(f(w))$, что
    $\mathcal{M}_2, w' \models \psi$. $f(w) R_2 w'$, значит найдется $u \in W_1$, что $f(u) = w'$ и $w R u$. По предположению
    индукции, $\mathcal{M}_1, u \models \psi$. Тогда $\mathcal{M}_1, w \models \Diamond \psi$.}
  \end{proof}
\end{frame}

\begin{frame}
  \frametitle{Лемма о $p$-морфизмах}

  \begin{lem}
    Пусть $f: \mathcal{F}_1 \twoheadrightarrow \mathcal{F}_2$, тогда $\operatorname{Log}(\mathcal{F}_1) \subseteq \operatorname{Log}(\mathcal{F}_2)$
  \end{lem}

\onslide<2->{
  \begin{proof}
    Пусть $\mathcal{F}_2 \nvDash \phi$. Тогда найдется такая оценка $\vartheta$ и $y \in W$,
    что в модели $\mathcal{M}_2 = \langle \mathcal{F}_2, \vartheta \rangle$ мы имеем $\mathcal{M}_2, y \nvDash \phi$. То есть,
    $\mathcal{M}_2, y \models \neg \phi$.}
\onslide<3->{
    Введем на $\mathcal{F}_1$ оценку $\varrho(p) = f^{-1}(\vartheta(p))$.
    В силу сюръективности $f$, найдется $x \in W_1$, что $f(x) = y$.}
    \onslide<4->{Откуда $\mathcal{M}_1, x \models \neg \phi$. Значит, $\mathcal{M}_1, x \nvDash \phi$.}
  \end{proof}

\end{frame}

\begin{frame}
  \frametitle{Невыразимость иррефлексивности в модальном языке}

\onslide<1->{
  \begin{lem}
    Не существует модальной формулы $\phi$, такой, что для любой шкалы $\mathcal{F}$ имеет место

    \begin{center}
      $\mathcal{F} \models \phi \Leftrightarrow \forall x \in W \neg (x R x)$
    \end{center}
  \end{lem}
  }

\onslide<2->{
  \begin{proof}
    Рассмотрим шкалы $\langle \mathbb{N}, < \rangle$ и $\langle \{ * \}, R = \{ (*,*)\} \rangle$.} \onslide<3->{Введем $f : \langle \mathbb{N}, < \rangle \to \langle \{ * \}, R \rangle$, где $f : n \mapsto *$.} \onslide<4->{Легко видеть, что
    $f$ монотонно и сюръективно:}\onslide<5->{ если $n < m$, то $f(n) = *$ и $f(m) = *$, а $*R*$.} \onslide<5->{Также $f$ обладает свойством поднятия.} \onslide<6->{Пусть $f(n) R *$ Положим $m := n + 1$. Тогда $n < m$ и $f(m) = *$.}
    \onslide<7->{По лемме о $p$-морфизмах, $\operatorname{Log}(\langle \mathbb{N}, < \rangle) \subseteq \operatorname{Log}(\langle \{ * \}, R \rangle)$. При этом первая шкала иррефлексивна, а вторая --- нет.}
  \end{proof}
\end{frame}


\begin{frame}
  \frametitle{На следующей лекции мы}

\onslide<1->{
  \begin{enumerate}
    \item Введем понятие (нормальной) модальной логики}
    \onslide<2->{\item Скажем, что такое полная по Крипке логика}
    \onslide<3->{\item Докажем теорему корректности, которая утверждает, что логика произвольного класса шкал является нормальной модальной логикой}
    \onslide<4->{\item Сформулируем минимальную нормальную модальную логику, логику ${\bf K}$}
    \onslide<5->{\item Докажем теорему полноты для ${\bf K}$: минимальная нормальная модальная логика является логикой класса всех шкал}
  \end{enumerate}
\end{frame}

\end{document}
