\documentclass[pdf,utf8,russian,aspectratio=169]{beamer}
\usepackage[T2A]{fontenc}
\usetheme{Copenhagen}
\usepackage{setspace}
\usepackage{amsmath}
\usepackage{pgfplots}
\usepackage[utf8]{inputenc}
\usepackage{tikz-cd}
\usepackage[all, 2cell]{xy}
\usepackage{amssymb}
\usepackage{verbatim}
\usepackage[all]{xy}
\usepackage{tikz}
\usepackage{bussproofs}
\usepackage{dsfont}
\usepackage{mathabx}
\usepackage{animate}
\usetikzlibrary{graphs}
\usetikzlibrary{arrows}
\usepackage{hyperref}
\usepackage[english,russian]{babel}
\usepackage{listings}
\usepackage{color}
\usepackage[all, 2cell]{xy}
\usepackage[all]{xy}
\usepackage{listings}
\usepackage{mathrsfs}
\newtheorem{defin}{Определение}
\newtheorem{theor}{Теорема}
\newtheorem{lem}{Лемма}
\title{Введение в модальную логику, \\ Лекция 1}
\author{Даня Рогозин \\ МГУ, Serokell}
\date{Computer Science Club}
\begin{document}

\maketitle

\begin{frame}
  \frametitle{Эпиграф}

  \begin{center}
    Необходимость непреклонна

\vspace{\baselineskip}

    Эврипид, 480-е --- 406 до н. э., \\
    финальные слова из трагедии "Гекуба"
  \end{center}
  \begin{center}
  \includegraphics[scale=0.25]{euripides.jpg}
  \end{center}
\end{frame}

\begin{frame}
  \frametitle{Что такое модальная логика?}

  В рамках курса мы дадим точное матетическое определение модальной логики. Пока что мы определим ее неформально.
  Неформально, модальная логика существует в двух смыслах:

  \begin{itemize}
    \item \emph{В широком смысле}, модальная логика является ветвью неклассической логики, которая изучает
    утверждения, дополненные специальными качествами, как доказуемость, необходимость, возможность, и так далее.
    Здесь модальность чаще всего понимается лингвистически, то есть как фигура речи, дополняющая предложение отдельным свойством. Например, "возможно, что завтра будет дождь" или "всегда было и будет, что $2 + 2 = 4$"
    \item \emph{В узком смысле}, под модальной логикой подразумевают конкретную логику, которая расширяет (как правило) классическую логику дополнительными модальными операторами. То есть, в узком смысле, модальная логика --- это некоторое отдельно взятое исчисление.
  \end{itemize}
\end{frame}

\begin{frame}
  \frametitle{Очень краткая история}

  \begin{itemize}
    \item Модальности в логике изучались еще античными греками, такими как Аристотель и Диодор Хронос
    \item В 17 веке Лейбниц предложил различать два вида истин: "истины" разума --- это те истины, которые верны всегда независимо от актуального положения дел; "истины" факта --- это истины, которые могли быть и неверны при прочем положении дел, и их "истинность" случайна
    \item Системы с модальностями впервые стал описывать Кларенс Льюис в начале 20-го века, который ввел в оборот связки $\Box$ и $\Diamond$, обозначающие необходимость и возможность соответственно. Тогда у этих модальностей не было семантики, связки понимались интуитивно
  \end{itemize}
\end{frame}

\begin{frame}
  \frametitle{Очень краткая история}
  \begin{itemize}
  \item В 1940-е годы некоторые из систем модальной логики изучались для описания топологических и метрических пространств. Это направление берет начало в работах Тарского и Маккинси
  \item В конце 1950-х годов Соул Крипке предложил строить реляционные модели для модальной и интуиционистской логик
  \item 1960-е--1970-е (что верно и по сей день) модальная логика эмансипируется от своих философских корней и становится математической самостоятельной дисциплиной, что сложилось в результате исследований Сегерберга, Салквиста, Гольдблатта, Томасона, Габбая, и т. д.
  \end{itemize}
\end{frame}

\begin{frame}
  \frametitle{Место модальной логики в общем контексте}

  В математической логике можно выделить три основных раздела:

  \begin{itemize}
    \item \emph{Теория доказательств} изучает свойства и структуру логических выводов, доказуемость в тех или иных формальных
    теориях
    \item \emph{Теория моделей} изучает семантику логических формул и их математическую интерпретацию
    \item \emph{Теория алгоритмов} ставит вопрос о вычислимости за конечное число шагов и о затратах ресурсов на выполнение той или иной процедуры.
  \end{itemize}
\end{frame}

\begin{frame}
  \frametitle{Место модальной логики в общем контексте}
  В таком делении модальная логика существует в трех испостасях:

  \begin{itemize}
    \item Синтаксис модальной логики, формулировка исчислений систем модальной логики, изучение вывода в модальной логике
    \item С точки зрения семантики, модальная логика служит довольно удобным аппаратом для характеризации математических структур, таких, как реляционные системы и топологические пространства
    \item С точки зрения теории алгоритмов, модальная логика изучается на предмет разрешимости и сложности вывода
  \end{itemize}

В рамках нашего курса мы будем изучать преимущественно второй и третий аспекты.
\end{frame}

\begin{frame}
  \frametitle{Модальный язык}

  \begin{defin}
    Пусть $\mathbb{V} = \{ p, q, r, ... \}$ счетное множество пропозициональных переменных. Множество формул $Fm$ определяется
    индуктивно:

    \begin{enumerate}
      \item Всякая переменная --- это формула
      \item Если $\phi$ --- это формула, то $\neg \phi$ --- это формула
      \item Если $\phi, \psi$ --- это формулы, то $(\phi \to \psi)$, $(\phi \land \psi)$, $(\phi \lor \psi)$ --- это формулы
      \item Если $\phi$ --- это формула, то $\Box \phi$ --- это формула
    \end{enumerate}

    По умолчанию, внешние скобки опускаются.

    Также нам будет у нас будет связка $\Diamond$, которую мы введем как сокращение $\Diamond \phi = \neg \Box \neg \phi$.
  \end{defin}
\end{frame}

\begin{frame}
  \frametitle{Модальный язык}
  Связки $\neg$, $\to$, $\land$ и $\lor$ мы понимаем классически, то есть как булевы отрицание, импликация, конъюнкция и
  дизъюнкция. Остается вопрос: как понимать $\Box$?

  Существует масса способов, вот некоторые из них:

  \begin{enumerate}
    \item $\Box \phi$ --- "$\phi$ необходимо". (алетическая модальность)
    \item $\Box \phi$ --- "$\phi$ известно". (эпистемическая модальность)
    \item $\Box \phi$ --- "утверждение $\phi$ доказуемо в арифметической теории $T$ (например, в арифметике Пеано)"
    \item $\Box \phi$ --- "внутренность множества $\phi$ в топологическом пространстве" (топологическая модальность, моя любимая)
    \item $\Box \phi$ --- "всегда будет $\phi$" (временная модальность)
  \end{enumerate}

   По умолчанию, мы как правило будет работать с модальностями как с абстрактными унарными операторами, но иногда будет обращаться к пунктами 1, 3 и 4.
\end{frame}

\begin{frame}
  \frametitle{Шкала Крипке}

  \begin{defin}
    Шкала Крипке --- это пара $\mathcal{F} = \langle W, R \rangle$, где $W$ непустое множество, а $R \subseteq W \times W$ --- это бинарное отношение на множестве $W$
  \end{defin}

\begin{itemize}
  \item Множество $W$ называют \emph{носителем шкалы}. Более философски, $W$ --- это множество возможных миров или положений дел. С временной точки зрения, $W$ --- это моменты времени.
  \item Отношение $R$ часто называют \emph{отношением достижимости}. Если $x$ находится $y$ в отношении $R$, то мы будем говорить, что "$x$ видит $y$" и писать $x R y$
  \item Можно также считать шкалу Крипке ориентированным графом, в котором $R$ --- это множество ребер в данном графе.
\end{itemize}

Также мы будет использовать обозначение $R(x) = \{ y \: | \: x R y \}$
\end{frame}

\begin{frame}
  \frametitle{Модель Крипке}
\begin{defin}
  Пусть $\mathcal{F} = \langle W, R \rangle$ --- это шкала Крипке. Модель Крипке --- это упорядоченная пара $\mathcal{M} = \langle \mathcal{F}, v \rangle$, где $\vartheta : \mathbb{V} \to 2^{W}$ --- это функция оценки.

  Отношение истинности (также, отношение вынуждения) $\mathcal{M}, w \models \phi$ определяется индукцией по формуле $\phi$:

  \begin{itemize}
    \item $\mathcal{M}, w \models p \Leftrightarrow w \in \vartheta(p)$
    \item $\mathcal{M}, w \models \neg \phi \Leftrightarrow \mathcal{M}, w\not\models \neg \phi$
    \item $\mathcal{M}, w \models \phi \lor \psi \Leftrightarrow \mathcal{M}, w \models \phi \text{ или } \mathcal{M}, w \models \psi$
    \item $\mathcal{M}, w \models \phi \land \psi \Leftrightarrow \mathcal{M}, w \models \phi \text{ и } \mathcal{M}, w \models \psi$
    \item $\mathcal{M}, w \models \Box \phi \Leftrightarrow \forall v \:\: w R v \Rightarrow \mathcal{M}, v \models \phi$
  \end{itemize}
\end{defin}

Из этого определения легко получить отношения истинности для импликации и ромба.
\end{frame}

\begin{frame}
  \frametitle{Модель Крипке}

  Неформально прокомментируем условие:
  \begin{itemize}
  \item $\mathcal{M}, w \models \Box \phi \Leftrightarrow \forall v \:\: w R v \Rightarrow \mathcal{M}, v \models \phi$
  \end{itemize}

  Здесь бокс удобнее всего читать через необходимость.

  Утверждение $\Box \phi$ истинно в мире $w$, если в любом мире $v$, достижимом из $w$, истинна формула $\phi$. То есть, $\phi$ истинно с необходимостью при положении дел $w$, если при любом развитии событий $v$, к которому можно прийти из $w$, утверждение $\phi$ будет истинно.

  \xymatrix{
  &&&& v_1 \models \phi \\
  &&& v_4 \models \phi & w \ar[u]^{R} \ar[d]_{R} \ar[l]_{R} \ar[r]^{R} \models \Box \phi & v_2 \models \phi \\
  &&&& v_3 \models \phi
  }
\end{frame}

\begin{frame}
  \frametitle{Пример модели Крипке}

\xymatrix{
&&&&&&& b \ar[dd]^{R} \ar@(ul,ur)^{R} \\
&&&&& a \ar[urr]^{R} \ar[drr]_{R} \\
&&&&&&& c \ar@(dr,dl)^{R} \\
}

Пусть $\vartheta(p) = \{ a, b, c \}$ и $\vartheta(q) = \{ b \}$.

Тогда
\begin{enumerate}
\item $a \models \Box p, \Box p \to \Box \Box p, \Diamond p \to \Diamond \Box p$.
\item $b \models \Box p \to p, \Box q \to q$
\item $c \models \Box p \to p$
\item $\mathcal{M} \models \Box p \to \Diamond p$
\end{enumerate}
\end{frame}

\begin{frame}
  \frametitle{Общезначимость в модели и истинность в шкале}

  \begin{defin}
    Формула $\phi$ \emph{истинна в модели} $\mathcal{M}$, если для любой точки $w \in W$, $\mathcal{M}, w \models \phi$.
    Кратко, $\mathcal{M} \models \phi$.
  \end{defin}

  \begin{defin}
    Формула $\phi$ \emph{истинна в шкале} $\mathcal{F}$, если для любой оценки $\vartheta$, в любой модели $\mathcal{M} = \langle \mathcal{F}, \vartheta \rangle$ формула $\phi$ истинна, то есть $\mathcal{M} \models \phi$.

    Обозначение $\mathcal{F} \models \phi$
  \end{defin}
\end{frame}

\begin{frame}
  \frametitle{Логика класса шкал}

  \begin{defin}
    Пусть $\mathcal{F}$ --- это шкала Крипке. \emph{Логика} шкалы $\mathcal{F}$ называется множество формул, истинных в данной шкале. Обозначение $Log(\mathcal{F}) = \{ \phi \in Fm \: | \: \mathcal{F} \models \phi \}$
  \end{defin}

  \begin{defin}
    Пусть $\mathbb{F}$ --- это класс шкал. \emph{Логикой класса шкал} $\mathbb{F}$ называется множество формул, истинных в каждой из шкал в данном классе. Обозначение $Log(\mathbb{F}) = \bigcap \limits_{\mathcal{F} \in \mathbb{F}} Log(\mathcal{F})$.
  \end{defin}
\end{frame}

\begin{frame}
  \frametitle{Операции над моделями и шкалами. $p$-морфизм шкал}
  \begin{defin} Ограниченным морфизмом шкал называется отображение $f : \mathcal{F}_1 \to \mathcal{F}_2$, такое что:
    \begin{enumerate}
      \item $f$ --- монотонно, $a R_1 b \Rightarrow f (a) R_2 f (b)$, где $a, b \in W_1$
      \item $f$ обладает свойством поднятия, то есть, $f(a) R_2 c \Rightarrow \exists b \in W_1 \: a R_1 b \: \& \: f(b) = c$:
      \begin{small}
      \xymatrix{
      &&&& a \ar@{|->}[d] \ar@{-->}[rr]^{R_1} && \exists b \ar@{|-->}[d] \\
      &&&& f(a) \ar[rr]_{R_2} && c
      }
    \end{small}
  \end{enumerate}

  Ограниченный морфизм $f : \mathcal{F}_1 \to \mathcal{F}_2$ называется $p$-морфизмом, если $f$ сюръективно, то есть $\forall y \in \mathcal{F}_2 \: \exists x \in \mathcal{F}_1 \:\: f(x) = y$. Обозначение $f : \mathcal{F}_1 \twoheadrightarrow \mathcal{F}_2$.
  \end{defin}
\end{frame}

\begin{frame}
  \frametitle{Операции над моделями и шкалами. $p$-морфизм шкал}
  \begin{defin}
    Органиченным морфизмом моделей Крипке $f : \mathcal{M}_1 = \langle \mathcal{F}_1, \vartheta_1 \rangle \to \mathcal{M}_2 \langle \mathcal{F}_2, \vartheta_2 \rangle$ называется ограниченный морфизм шкал $f : \mathcal{F}_1 \to \mathcal{F}_2$ со следующим условием:

    \begin{center}
      $\mathcal{M}_1, w \models p \Leftrightarrow \mathcal{M}_1, f(w) \models p$, для всех $w \in \mathcal{M}_1$, где $p$ --- это переменная.
    \end{center}

    Аналогично, ограниченный морфизм моделей называется $p$-морфизмом моделей, если отображение $f$ --- сюръективно.
  \end{defin}

  Ограниченные морфизмы позволяют отображать модели и шкалы, сохраняя истинность формул. Сформулируем лемму:
\end{frame}

\begin{frame}
  \frametitle{Операции над моделями и шкалами. Лемма о $p$-морфизмах}

  \begin{lem}
    \begin{enumerate}
      \item Пусть $\mathcal{M}_1, \mathcal{M}_2$ --- модели Крипке и $f : \mathcal{M}_1 \to \mathcal{M}_2$. Тогда
      $\mathcal{M}_1, w \models \phi \Leftrightarrow \mathcal{M}_2, f(w) \models \phi$
      \item Пусть $f : \mathcal{M}_1 \twoheadrightarrow \mathcal{M}_2$. Тогда
      $\mathcal{M}_1 \models \phi \Leftrightarrow \mathcal{M}_2 \models \phi$
      \item Пусть $f: \mathcal{F}_1 \twoheadrightarrow \mathcal{F}_2$, тогда $Log(\mathcal{F}_1) \subseteq Log(\mathcal{F}_2)$
    \end{enumerate}
  \end{lem}
\end{frame}

\begin{frame}
  \frametitle{Операции над моделями и шкалами. Лемма о $p$-морфизмах}

  Докажем первый пункт

  \begin{lem}
    Пусть $\mathcal{M}_1, \mathcal{M}_2$ --- модели Крипке и $f : \mathcal{M}_1 \to \mathcal{M}_2$. Тогда
    $\mathcal{M}_1, w \models \phi \Leftrightarrow \mathcal{M}_2, f(w) \models \phi$
  \end{lem}

Индукция по построению формулы $\phi$. Рассмотрим случай, когда $\phi \eqcirc \Box \psi$. Покажем, что $\mathcal{M}_1, w \models \Box \psi \Leftrightarrow \mathcal{M}_2, f(w) \models \Box \psi$

  \begin{proof}
    ($\Rightarrow$) Пусть $f : \mathcal{M}_1 \to \mathcal{M}_2$. Пусть $\mathcal{M}_1, w \models \Box \psi$. Пусть $v' \in W_2$, такой, что $f(w) R_2 v'$, тогда, по свойству поднятия, найдется такой $v \in W_1$, то $w R_1 v$ и $f(v) = v'$. Так как $w R_1 v$, тогда $\mathcal{M}_1, v \models \psi$. Но по предположению индукции, $\mathcal{M}_2, f(v) \models \psi$, то есть $\mathcal{M}_2, v' \models \psi$.
  \end{proof}
\end{frame}

\begin{frame}
  \frametitle{Операции над моделями и шкалами. Лемма о $p$-морфизмах}

  Докажем первый пункт

  \begin{lem}
    Пусть $\mathcal{M}_1, \mathcal{M}_2$ --- модели Крипке и $f : \mathcal{M}_1 \to \mathcal{M}_2$. Тогда
    $\mathcal{M}_1, w \models \phi \Leftrightarrow \mathcal{M}_1, f(w) \models \phi$
  \end{lem}

  Индукция по построению формулы $\phi$. Рассмотрим случай, когда $\phi \eqcirc \Box \psi$. Покажем, что $\mathcal{M}_1, w \models \Box \psi \Leftrightarrow \mathcal{M}_2, f(w) \models \Box \psi$

  \begin{proof}
    ($\Leftarrow$) Пусть $\mathcal{M}_2, f(w) \models \Box \psi$. Пусть $v \in W_1$, т.ч., $w R_1 v$. По монотонности,
    $f(w) R_2 f(v)$. Тогда $\mathcal{M}_2, f(v) \models \phi$, а также $\mathcal{M}_1, v \models \phi$ по предположению индукции. Тогда $\mathcal{M}_1, w \models \Box \psi$.
  \end{proof}
\end{frame}

\begin{frame}
  \frametitle{Операции над моделями и шкалами. Лемма о $p$-морфизмах}

  Доказательство второго пунтка

  \begin{lem}
    Пусть $f : \mathcal{M}_1 \twoheadrightarrow \mathcal{M}_2$. Тогда $\mathcal{M}_1 \models \phi \Leftrightarrow \mathcal{M}_2 \models \phi$
  \end{lem}

  \begin{proof}
    ($\Rightarrow$) Пусть $\mathcal{M}_1 \models \phi$ и $w' \in W_2$. Так как $f$ сюръективно, найдется такой $w \in W_1$,
    что $f(w) = w'$. Так как $\mathcal{M}_1 \models \phi$, то $\mathcal{M}_1, w \models \phi$, тогда по предыдущему пункту
    $\mathcal{M}_2, f(w) \models \phi$, то есть, $\mathcal{M}_2, w' \models \phi$


    ($\Leftarrow$) Пусть $\mathcal{M}_2 \models \phi$, тогда $\mathcal{M}_2, f(w) \models \phi$, и по первому пункту, $\mathcal{M}_1, w \models \phi$
  \end{proof}
\end{frame}

\begin{frame}
  \frametitle{Операции над моделями и шкалами. Лемма о $p$-морфизмах}

  \begin{lem}
    Пусть $f: \mathcal{F}_1 \twoheadrightarrow \mathcal{F}_2$, тогда $Log(\mathcal{F}_1) \subseteq Log(\mathcal{F}_2)$
  \end{lem}

  \begin{proof}
    Пусть $\mathcal{F}_1 \models \phi$. Пусть $\mathcal{M}_2 = \langle \mathcal{F}_2, \vartheta_2 \rangle$ --- модель на шкале
    $\mathcal{F}_2$. Посмотрим модель $\mathcal{M}_1 = \langle \mathcal{F}_1, \vartheta_1 \rangle$ и положим $\mathcal{M}_1, w
    \models p \Leftrightarrow \mathcal{M}_1, f(w) \models p$. Тогда $\mathcal{M}_1 \twoheadrightarrow \mathcal{M}_2$ и, по
    предыдущему пункту, $\mathcal{M}_2 \models \phi$.
  \end{proof}

\end{frame}

\begin{frame}
  \frametitle{Операции над моделями и шкалами. Порожденная подмодель}

  \begin{defin} Пусть $\mathcal{F} = \langle W, R \rangle$ --- это шкала Крипке и
  $V \subseteq W$ --- это непустое подмножество $W$. Тогда сужение шкалы --- это шкала
  $\mathcal{F} \upharpoonright V = \langle V, R \upharpoonright V \rangle$, где $R \upharpoonright V = R \cap V \times V$.

  \end{defin}

  \begin{defin}
    Пусть $\mathcal{M} = \langle \mathcal{F}, \vartheta \rangle$ --- это модель Крипке и $V \subseteq W$ --- это непустое подмножество $W$. Тогда сужение модели на $V$ --- это мдель $\mathcal{M} \upharpoonright V = \langle \mathcal{F} \upharpoonright V, \vartheta' \rangle$, где $\mathcal{F} \upharpoonright V$ --- это сужение шкалы на $V$ и $\vartheta'(p) = \vartheta(p) \cap V$ для каждой переменной $p$.
  \end{defin}

  Подмножество $V$ $R$-замкнуто, если $x \in V$ и $x R y$ влечет $y \in V$. В таком случае сужение модели и шкалы называются порожденными подшкалой и подмоделью.
\end{frame}

\begin{frame}
  \frametitle{Операции над моделями и шкалами. Лемма о порожденной подмодели}

\begin{lem} Пусть $\mathcal{F} = \langle W, R \rangle$ --- это шкала и $\mathcal{M} = \langle \mathcal{F}, \vartheta \rangle$
  --- это модель Крипке. Пусть также $V \subseteq W$ $R$-замкнуто, тогда
  \begin{enumerate}
    \item Если $w \in V$, тогда $\mathcal{M}, w \models \phi \Leftrightarrow \mathcal{M} \upharpoonright V, w \models \phi $
    \item $Log(\mathcal{F}) \subseteq Log(\mathcal{F} \upharpoonright V)$
  \end{enumerate}
\end{lem}

Докажем первый пункт

\begin{proof}
  ($\Rightarrow$) Пусть $\mathcal{M}, w \models \Box \phi$, так как $w \in V$ и $V$ $R$-замкнуто, для любого $v \in R(w)$,
  $\langle w, v \rangle \in R \upharpoonright V$. По предположению индукции, $\mathcal{M} \upharpoonright V, v \models \phi$.
  Тогда $\mathcal{M} \upharpoonright V, w \models \Box \phi$.
\end{proof}
\end{frame}

\begin{frame}
  \frametitle{Операции над моделями и шкалами. Лемма о порожденной подмодели}

\begin{lem} Пусть $\mathcal{F} = \langle W, R \rangle$ --- это шкала и $\mathcal{M} = \langle \mathcal{F}, \vartheta \rangle$
  --- это модель Крипке. Пусть также $V \subseteq W$ $R$-замкнуто, тогда
  \begin{enumerate}
    \item Если $w \in V$, тогда $\mathcal{M}, w \models \phi \Leftrightarrow \mathcal{M} \upharpoonright V, w \models \phi $
    \item $Log(\mathcal{F}) \subseteq Log(\mathcal{F} \upharpoonright V)$
  \end{enumerate}
\end{lem}

Докажем первый пункт

\begin{proof}
  ($\Leftarrow$) Пусть $\mathcal{M} \upharpoonright V, w \models \Box \phi$.
  Тогда $\mathcal{M} \upharpoonright V, v \models \phi$ и $\langle w, v \rangle \in R \upharpoonright V$ для любого $v$.
  По предположению индукции, $\mathcal{M}, v \models \phi$, тогда $\mathcal{M}, w \models \Box \phi$
\end{proof}
\end{frame}

\begin{frame}
  \frametitle{Операции над моделями и шкалами. Лемма о порожденной подмодели}

\begin{lem} Пусть $\mathcal{F} = \langle W, R \rangle$ --- это шкала и $\mathcal{M} = \langle \mathcal{F}, \vartheta \rangle$
  --- это модель Крипке. Пусть также $V \subseteq W$ $R$-замкнуто, тогда
  \begin{enumerate}
    \item Если $w \in V$, тогда $\mathcal{M}, w \models \phi \Leftrightarrow \mathcal{M} \upharpoonright V, w \models \phi $
    \item $Log(\mathcal{F}) \subseteq Log(\mathcal{F} \upharpoonright V)$
  \end{enumerate}
\end{lem}

Докажем второй пункт
\begin{proof} Предположим $\mathcal{F} \upharpoonright V\not\models \phi$. Тогда найдется модель $\mathcal{M}_2 = \langle
\mathcal{F} \upharpoonright V, \vartheta' \rangle$ и $w \in W$, такие, что $\mathcal{M}_2, w\not\models \phi$, тогда
$\mathcal{M}_2, w \models \neg\phi$. Рассмотрим модель $\mathcal{M}_1 = \langle \mathcal{F}, \vartheta \rangle$ для $\vartheta(p) = \vartheta'(p)$ для всех переменных $p$. По предыдущему пункту, $\mathcal{M}_1, w \models \neg\phi$, тогда
$\mathcal{M}_1, w\not\models \phi$.
\end{proof}

\end{frame}

\begin{frame}
  \frametitle{На следующей лекции мы}

  \begin{enumerate}
    \item Введем понятие (нормальной) модальной логики
    \item Введем понятие полной по Крипке логики
    \item Докажем теорему корректности, которая утверждает, что логика произвольного класса шкал является модальной логикой
    \item Сформулируем минимальную нормальную модальную логику, систему ${\bf K}$
    \item Докажем теорему полноты для ${\bf K}$: минимальная нормальная модальная логика является логикой класса всех шкал
    \item Рассмотрим примеры модальных формул, охарактеризуем соответствующие им свойства шкал и опишем стандартный "зоопарк"
    модальных логик: системы ${\bf T}$, ${\bf K4}$, ${\bf S}4$, ${\bf S}5$, ${\bf D}$, ${\bf GL}$, ${\bf Grz}$, и так далее.
  \end{enumerate}
\end{frame}

\end{document}
