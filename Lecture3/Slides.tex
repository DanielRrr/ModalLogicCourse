\documentclass[pdf,utf8,russian,aspectratio=169]{beamer}
\usepackage[T2A]{fontenc}
\usetheme{Copenhagen}
\usepackage{setspace}
\usepackage{amsmath}
\usepackage{pgfplots}
\usepackage[utf8]{inputenc}
\usepackage{tikz-cd}
\usepackage[all, 2cell]{xy}
\usepackage{amssymb}
\usepackage{verbatim}
\usepackage[all]{xy}
\usepackage{tikz}
\usepackage{bussproofs}
\usepackage{dsfont}
\usepackage{mathabx}
\usepackage{animate}
\usetikzlibrary{graphs}
\usetikzlibrary{arrows}
\usepackage{hyperref}
\usepackage[english,russian]{babel}
\usepackage{listings}
\usepackage{color}
\usepackage[all, 2cell]{xy}
\usepackage[all]{xy}
\usepackage{listings}
\usepackage{mathrsfs}
\newtheorem{defin}{Определение}
\newtheorem{theor}{Теорема}
\newtheorem{lem}{Лемма}
\newtheorem{col}{Следствие}
\title{Введение в модальную логику, \\ Лекция 3}
\author{Даня Рогозин \\ МГУ, Serokell}
\date{20 октября \\ Computer Science Club \\ ПОМИ РАН}
\begin{document}

\maketitle

\begin{frame}
  \frametitle{На прошлой лекции мы}

  \begin{itemize}
    \item Определили, что такое нормальная модальная логика
    \item Сформулировали минимальную нормальную модальную логику, логику ${\bf K}$
    \item Ввели каноническую модель, доказали лемму Линденбаума и лемму о канонической модели
    \item Показали, что ${\bf K}$ является логикой класса всех шкал Крипке
  \end{itemize}
\end{frame}


\begin{frame}
  \frametitle{Каноническая шкала и каноническая модель}

  Напомним определения канонической шкалы и канонической модели
\begin{itemize}
\item \emph{Канонической шкалой} логики $\mathcal{L}$ называется пара $\mathcal{F}_{\mathcal{L}} = \langle W_{\mathcal{L}}, \mathcal{W}_{\mathcal{R}} \rangle$, где
\begin{enumerate}
  \item $W_{\mathcal{L}}$ --- это множество всех максимальных $\mathcal{L}$-непротиворечивых множеств
  \item $\Gamma R_{\mathcal{L}} \Delta \Leftrightarrow \forall \phi \:\: \Box \phi \in \Gamma \Rightarrow \phi \in \Delta$
  \end{enumerate}
  \item Также условие на отношение достижимости в канонической шкале можно ввести так:
  \begin{center}
    $\forall \phi \:\: (\phi \in \Delta \Rightarrow \Diamond \phi \in \Gamma)$
  \end{center}
  \item \emph{Канонической моделью} мы называли модель $\mathcal{M}_{\mathcal{L}}$, состоящей из канонической шкалы и канонической оценки, то есть такой оценки $\vartheta_{\mathcal{L}}$, что
  \begin{center}
    $\vartheta_{\mathcal{L}}(p) = \{ \Gamma \in W_{\mathcal{L}} \: | \: p \in \Gamma \}$
  \end{center}
\end{itemize}
\end{frame}

\begin{frame}
  \frametitle{Лемма о канонической модели}
  Вспомним, что мы доказали про максимально непротиворечивые множества и каноническую модель:
\begin{enumerate}
  \item Всякое $\mathcal{L}$-непротиворечивое множество содержится в максимальном $\mathcal{L}$-непротиворечивом (лемма Линденбаума)
  \item Если $\Gamma$ максимальное $\mathcal{L}$-непротиворечивое множество, то
  \begin{itemize}
    \item $\mathcal{L} \subseteq \Gamma$
    \item $\phi \in \Gamma \Leftrightarrow \neg \phi \notin \Gamma$
    \item $\phi \land \psi \in \Gamma \Leftrightarrow \phi, \psi \in \Gamma$
  \end{itemize}
  \item $\mathcal{M}_{\mathcal{L}}, \Gamma \models \phi \Leftrightarrow \phi \in \Gamma$
  \item $\mathcal{L} \vdash \phi \Leftrightarrow \mathcal{M}_{\mathcal{L}} \models \phi$
\end{enumerate}
\end{frame}

\begin{frame}
  \frametitle{Модальные логики}

  Вспомним к список модальных формул из первой лекции:

  \begin{itemize}
    \item {\bf AT} $\Box p \to p$ (рефлексивность)
    \item {\bf A4} $\Box p \to \Box \Box p$ (транзитивность)
    \item {\bf AB} $p \to \Box \Diamond p$ (рефлексивность)
    \item {\bf ACR} $\Diamond \Box p \to \Box \Diamond p$ (формула Черча-Россера)
    \item {\bf AD} $\Diamond \top$ (сериальность)
  \end{itemize}

  Определим список логик:
\end{frame}

\begin{frame}
  \frametitle{Модальные логики}

  Примеры модальных логик:

\begin{itemize}
  \item ${\bf T} = {\bf K} \oplus \Box p \to p$
  \item ${\bf K}4 = {\bf K} \oplus \Box p \to \Box \Box p$
  \item ${\bf D} = {\bf K} \oplus \Diamond \top$
  \item ${\bf S}4 = {\bf K}4 \oplus \Box p \to p = {\bf T} \oplus \Box p \to \Box \Box p$
  \item ${\bf S}5 = {\bf S}4 \oplus p \to \Box \Diamond p$
  \item ${\bf S}4.2 = {\bf S}4 \oplus \Diamond \Box p \to \Box \Diamond p$
\end{itemize}
\end{frame}

\begin{frame}
  \frametitle{Каноническая логика}

  \begin{defin}
    \begin{itemize}
    \item Формула $\phi$ называется канонической, если логика $\mathcal{L} = {\bf K} \oplus \phi = Log(\mathcal{F}_{\mathcal{L}})$
    \item Логика называется канонической, если $\mathcal{L} = Log(\mathcal{F}_{\mathcal{L}})$.
  \end{itemize}
  \end{defin}

  \begin{lem}
    Если логика каноническая, то она полна.
  \end{lem}
\end{frame}


\begin{frame}
  \frametitle{Лемма о каноничности формул}
  \begin{lem}
    Следующие формулы канонические:
    \begin{enumerate}
      \item {\bf AT} $\Box p \to p$
      \item {\bf A4} $\Box p \to \Box \: \Box p$
      \item {\bf AB} $p \to \Box \Diamond p$
      \item {\bf ACR} $\Diamond \Box p \to \Box \Diamond p$
      \item {\bf AD} $\Diamond \top$
    \end{enumerate}
  \end{lem}

  Рассмотрим случаи ${\bf A}4$ и ${\bf AT}$.
\end{frame}

\begin{frame}
  \frametitle{Лемма о каноничности формул}

  \begin{lem}
    Формула $\Box p \to \Box \: \Box p$ канонична, то есть $\mathcal{F}_{{\bf K} 4} \models \Box p \to \Box \Box p$.
  \end{lem}

  Снова перепишем формулу ${\bf A}4$ как $\Diamond \Diamond p \to \Diamond p$

  \begin{proof}
    Пусть $\Gamma, \Delta, \Theta \in W_{{\bf K}4}$, такие что $\Gamma R_{{\bf K} 4} \Delta$ и $\Delta R_{{\bf K} 4} \Theta$. Покажем, что если $\phi \in \Theta$, то $\Diamond \phi \in \Gamma$.
    Пусть $\phi \in \Theta$, тогда, по лемме о канонической модели, $\mathcal{M}_{{\bf K}4}, \Theta \models \phi$. Тогда
    $\mathcal{M}_{{\bf K}4}, \Delta \models \Diamond \phi$, откуда $\mathcal{M}_{{\bf K}4}, \Gamma \models \Diamond \Diamond
    \phi$. С другой стороны, $\mathcal{M}_{{\bf K}4}, \Gamma \models \Diamond \Diamond \phi \to \Diamond \phi$ по лемме о
    канонической модели. Тогда $\mathcal{M}_{{\bf K}4}, \Gamma \models \Diamond \phi$, тогда $\Diamond \phi \in \Gamma$.
  \end{proof}
\end{frame}

\begin{frame}
  \frametitle{Лемма о каноничности формул}

  \begin{lem}
    Формула $\Box p \to p$ канонична, то есть $\mathcal{F}_{T} \models \Box p \to p$.
  \end{lem}

\begin{proof}
  Пусть $\Gamma \in W_{\bf T}$ и $\mathcal{M}_{\bf T}, \Gamma \models p$.
  $\mathcal{M}_{\bf T}, \Gamma \models p \to \Diamond p$ по лемме о канонической модели и
  $\mathcal{M}_{\bf T}, \Gamma \models \Diamond p$.
  Тогда $\Diamond p \in \Gamma$, значит, $\Gamma R_{\bf T} \Gamma$.
\end{proof}
\end{frame}

\begin{frame}
  \frametitle{Лемма о каноничности логик}

Вернемся к списку логик:

  \begin{itemize}
    \item ${\bf T} = {\bf K} \oplus \Box p \to p$
    \item ${\bf K}4 = {\bf K} \oplus \Box p \to \Box \Box p$
    \item ${\bf D} = {\bf K} \oplus \Diamond \top$
    \item ${\bf S}4 = {\bf K}4 \oplus \Box p \to p = {\bf T} \oplus \Box p \to \Box \Box p$
    \item ${\bf S}5 = {\bf S}4 \oplus p \to \Box \Diamond p$
    \item ${\bf S}4.2 = {\bf S}4 \oplus \Diamond \Box p \to \Box \Diamond p$
  \end{itemize}

Следующее утверждение легко следует из леммы о каноничности формул:


\begin{lem}
  Логики из списка выше являются каноническими
\end{lem}

\end{frame}

\begin{frame}
  \frametitle{Теорема о полноте}

  Имеет место следующая теорема:

\begin{theor}
  \begin{enumerate}
    \item ${\bf T} = Log(\mathbb{F})$, где $\mathbb{F}$ --- класс всех рефлексивных шкал.
    \item ${\bf K}4 = Log(\mathbb{F})$, где $\mathbb{F}$ --- класс всех транзитивных шкал.
    \item ${\bf D} = Log(\mathbb{F})$, где $\mathbb{F}$ --- класс всех сериальных шкал.
    \item ${\bf S}4 = Log(\mathbb{F})$, где $\mathbb{F}$ --- класс всех предпорядков.
    \item ${\bf S}5 = Log(\mathbb{F})$, где $\mathbb{F}$ --- класс всех отношений эквивалентности.
    \item ${\bf S}4.2 = Log(\mathbb{F})$, где $\mathbb{F}$ --- класс всех конфлюентных предпорядков.
  \end{enumerate}
\end{theor}

\begin{proof}
  Рассмотрим в качестве примера логику ${\bf S}4$
\end{proof}
\end{frame}

\begin{frame}
  \frametitle{Теорема о полноте}

\begin{theor}
    ${\bf S}4 = Log(\mathbb{F})$, где $\mathbb{F}$ --- класс всех предпорядков.
\end{theor}

\begin{proof}
    Рассмотрим следующие включения:

\begin{center}
    ${\bf S}4 \subseteq Log(\mathbb{F}) \subseteq Log(\mathcal{F}_{{\bf S}4}) \subseteq {\bf S}4$
\end{center}


Первое включение --- это теорема корректности для ${\bf S}4$.

Третье включение (и, более того, равенство) следует из того факта, что ${\bf S}4$ является канонической логикой.

Второе включение следует из того факта, каноническая шкала $\mathcal{F}_{{\bf S}4}$ является предпорядком.
\end{proof}


\end{frame}

\begin{frame}
  \frametitle{Пример неканонической логики}

  Мы сказали, что всякая каноническая логика полна. Рассмотрим пример логики, которая полна по Крипке, но не является канонической.
\end{frame}

\begin{frame}
  \frametitle{Н\"{e}терово отношение}
\begin{defin}
  Отношение $R \subseteq W \times W$ является обратно фундированным (н\"{e}теровым), если не существует бесконечных цепей $a_1 R a_2 R \dots$.
\end{defin}
\end{frame}

\begin{frame}
  \frametitle{Эквивалентные определения н\"{e}теровости}
\begin{lem}
  Отношение $R \subseteq W \times W$ является н\"{е}теровым $\Leftrightarrow$ каждое непустое подмножество $W$ имеет $R$-максимальный элемент.
\end{lem}

\begin{proof}
  ($\Leftarrow$) Пусть отношение $R$ не является н\"{е}теровым, тогда найдется бесконечная цепь $a_1 R a_2 R \dots$. Тогда множество $W' = \{ a_1, a_2, \dots \}$ не будет содержать максимального элемента.
\end{proof}
\end{frame}

\begin{frame}
  \frametitle{Эквивалентные определения н\"{e}теровости}
\begin{lem}
  Отношение $R \subseteq W \times W$ является н\"{е}теровым $\Leftrightarrow$ каждое непустое подмножество $W$ имеет $R$-максимальный элемент.
\end{lem}

WARNING: обратная импликация требует аксиомы выбора.

\begin{proof}
  ($\Rightarrow$) Пусть $W'$ не имеет максимального элемента. Пусть $a_0 \in W'$, тогда найдется $a_1$, такой, что $a_0 R a_1$. Дальше, для любого $n \in \mathbb{N}$, $a_n R a_{n + 1}$. По аксиоме выбора, можно построить последовательность $\{ a_n \}_{n \in \mathbb{N}}$, которая по отношению $R$ и образует
  бесконечную последовательность $a_0 R a_1 R a_2 \dots$.
\end{proof}
\end{frame}

\begin{frame}
  \frametitle{Формула Л\"{е}ба}

\begin{defin}
    Формулой Л\"{е}ба называется формула вида $\Box (\Box p \to p) \to \Box p$, или, что эквивалентно, $\Diamond p \to \Diamond (p \land \neg \Diamond p)$.
\end{defin}

\begin{defin} Логика Г\"{е}деля-Л\"{е}ба
  ${\bf GL} = {\bf K} \oplus \Box (\Box p \to p) \to \Box p$
\end{defin}


\end{frame}

\begin{frame}
  \frametitle{Н\"{e}теровость шкал}
\begin{lem}
  Пусть $\mathcal{F} = \langle W, R \rangle$ --- это шкала Крипке, тогда

\begin{itemize}
  \item $\mathcal{F} \models \Box (\Box p \to p) \to \Box p \Leftrightarrow R \text{ транзитивно и н\"{е}терово}$
\end{itemize}
\end{lem}

\begin{proof}
  $(\Rightarrow)$. Покажем транзитивность.

  Пусть $x, y, z \in W$ и $x R y R z$. Положим $\vartheta(p) = W \setminus \{ y, z \}$ и $\mathcal{M} = \langle \mathcal{F}, \vartheta \rangle$.
  Тогда $\mathcal{M}, x \nvDash \Box p$ и $\mathcal{M}, y \nvDash p$. При этом, ясно, что $\mathcal{M}, x \models \Box (\Box p \to p) \to \Box p$.

  Тогда $\mathcal{M}, x \nvDash \Box (\Box p \to p)$, тогда найдется $z' \in R(x)$, что $\mathcal{M}, z' \models \Box p$ и $\mathcal{M}, z' \nvDash p$.

  Значит, $z' = y$ или $z' = z$. Первое равенство неверно, так как $\mathcal{M}, y \nvDash p$. Остается только $z$, тогда $x R z$.
\end{proof}
\end{frame}

\begin{frame}
  \frametitle{Н\"{e}теровость шкал}
\begin{lem}
  Пусть $\mathcal{F} = \langle W, R \rangle$ --- это шкала Крипке, тогда

\begin{itemize}
  \item $\mathcal{F} \models \Box (\Box p \to p) \to \Box p \Leftrightarrow R \text{ транзитивно и н\"{е}терово}$
\end{itemize}
\end{lem}

\begin{proof}
  $(\Rightarrow)$. Покажем н\"{e}теровость.

  Пусть $W' \subseteq W$ непустое подмножество и $x \in W'$. Положим $\vartheta(p) = W'$, покажем, что $W'$ имеет максимальный элемент. Если $x$ не максимальный, то $\mathcal{M}, x \models \Diamond p$. Тогда $\mathcal{M}, x \models \Diamond (p \land \neg \Diamond p)$. То есть, найдется $y \in R(x)$, что $\mathcal{M}, y \models p \land \neg \Diamond p$. Тогда $у$ максимальный элемент $W'$.

\end{proof}
\end{frame}

\begin{frame}
  \frametitle{Н\"{e}теровость шкал}

  \begin{lem}
    Пусть $\mathcal{F} = \langle W, R \rangle$ --- это шкала Крипке, тогда

  \begin{itemize}
    \item $\mathcal{F} \models \Box (\Box p \to p) \to \Box p \Leftrightarrow R \text{ транзитивно и н\"{е}терово}$
  \end{itemize}
\end{lem}

\begin{proof}
  $(\Leftarrow)$ Пусть $\mathcal{F}$ транзитивна и н\"{e}терова и $\vartheta$ --- это оценка. Пусть $x \in W$ и $\mathcal{M}, x \models \Diamond p$.
  Положим $W' = \vartheta(p) \cap R(x)$, которое непусто. Пусть $y$ --- максимальный элемент $W'$, тогда $\mathcal{M}, y \models p$ и
  $\mathcal{M}, y \nvDash \Diamond p$, то есть $\mathcal{M}, y \models \neg \Diamond p$.
  Тогда $\mathcal{M}, y \models p \land \neg \Diamond p$.
  Откуда $\mathcal{M}, x \models \Diamond p \to \Diamond (p \land \neg \Diamond p)$.
\end{proof}
\end{frame}

\begin{frame}
  \frametitle{Н\"{e}теровость шкал}
  \begin{col}
    \begin{enumerate}

    \item Если $\mathcal{F} \models \Box (\Box p \to p) \to \Box p$, тогда $\mathcal{F}$ иррефлексивна, то есть $\forall x \in W \:\: \neg (x R x)$

    \item Если $\mathcal{F} \models \Box (\Box p \to p) \to \Box p$, тогда $\mathcal{F} \nvDash \Box p \to p$
  \end{enumerate}
  \end{col}

  \begin{proof}
    \begin{enumerate}
    \item Пусть $\mathcal{F} \models \Box (\Box p \to p) \to \Box p$ и существует $x \in W$, такой что $x R x$.
    Данная шкала является н\"{e}теровой.
    С другой стороны, существует бесконечно возрастающая цепь $x R x R x \dots$, противоречие.
    \item Очевидно.
  \end{enumerate}
  \end{proof}
\end{frame}

\begin{frame}
  \frametitle{Неканоничность формулы Л\"{е}ба}

  \begin{lem}
    $\mathcal{F}_{\bf GL} \nvDash \Box (\Box p \to p) \to \Box p$
  \end{lem}

  \begin{proof}
    Покажем, что отношение $R_{\bf GL}$ в канонической шкале не является иррефлексивным. Пусть $\Gamma = \{ \phi \to \Diamond \phi \: | \: \phi \in Fm \}$. Покажем, что $\Gamma$ ${\bf GL}$-
    непротиворечиво. Предположим противное. Тогда найдутся такие $\psi_1, \dots, \psi_n$, что
    $\neg ((\psi_1 \to \Diamond \psi_1) \land \dots \land (\psi_n \to \Diamond \psi_n)) \in {\bf GL}$.
    Данная формула эквивалентна следующей
    \begin{center}
    $ (\psi_1 \land \Box \neg \psi_1) \lor \dots \lor (\psi_n \land \Box \neg \psi_n)) \in {\bf GL}$.
    \end{center}
    Обозначим последнюю формулу как $\psi$.

  \end{proof}
\end{frame}

\begin{frame}
  \frametitle{Неканоничность формулы Л\"{е}ба}

  \begin{lem}
    $\mathcal{F}_{\bf GL} \nvDash \Box (\Box p \to p) \to \Box p$
  \end{lem}

  \begin{proof}
    (Продолжение.) Рассмотрим шкалу $\mathcal{F} = \langle n + 1, < \rangle$, где $n + 1 = \{ 0, \dots, n \}$.
    Легко видеть, что $\mathcal{F} \models {\bf GL}$. Пусть $\mathcal{M}$ --- это модель на шкале $\mathcal{F}$.
    Тогда $\mathcal{M} \models \psi$. Тогда в каждой точке модели истинен какой-то дизъюнктивный член
    $\psi_i \land \Box \neg \psi_i$ при некотором $i \in \{ 1, \dots, n\}.$
    С другой стороны, у нас $n$ дизъюнктивных членов, тогда как в модели $n + 1$ точка.
  \end{proof}
\end{frame}

\begin{frame}
  \frametitle{Неканоничность формулы Л\"{е}ба}

  \begin{lem}
    $\mathcal{F}_{\bf GL} \nvDash \Box (\Box p \to p) \to \Box p$
  \end{lem}

  \begin{proof}
    (Продолжение.) Тогда найдется такое $j \in \{ 1, \dots, n\}$, что $\mathcal{M}, k \models \psi_j \land \Box \neg \psi_j$ и $\mathcal{M}, l \models \psi_j \land \Box \neg \psi_j$, при некоторых $k, l \in n + 1$ и $k < l$.
    Тогда $\mathcal{M}, k \models \Diamond \psi_j$.
    При этом, $\mathcal{M}, k \models \Box \neg \psi_j$, что эквивалентно, $\mathcal{M}, k \models \neg \Diamond \psi_j$.
    Противоречие.

    Тогда $\Gamma$ ${\bf GL}$-непротиворечиво.
    По лемме Линденбаума, множество $\Gamma$ содержится в некотором максимально ${\bf GL}$-непротиворечивом множестве $\Delta$.
    Нетрудно проверить, что $\Delta R_{\bf GL} \Delta$.
  \end{proof}
\end{frame}

\begin{frame}
  \frametitle{Что делать?}

\begin{itemize}
  \item Таким образом, логика Г\"{е}деля-Л\"{е}ба не является логикой собственной канонической шкалы.
  \item То есть, мы не можем доказать ее полноту способом аналогичным тому, что мы использовали, скажем, для логики предпорядков ${\bf S}4$.
  \item Полноту логики ${\bf GL}$ относительно транзитивных н\"{е}теровых шкал можно доказать, но чуть более витеиватым способом.
  \item Сначала поговорим о селективной фильтрации модели
\end{itemize}
\end{frame}

\begin{frame}
  \frametitle{Селективная фильтрация}
  \begin{defin}
    Пусть $\mathcal{M} = \langle W, R, \vartheta \rangle$ и $\Gamma$ --- множество формул, замкнутое относительно подформул.
    Селективной фильтрацией модели $\mathcal{M}$ по множеству формул $\Gamma$ называется модель $\mathcal{M} = \langle W', R', \vartheta' \rangle$, где
    $W' \subseteq W$, $R' \subseteq R$ и $\vartheta'(p) = \vartheta(p) \cap W'$ для $p \in \Gamma$ со следующим условием:

    \begin{itemize}
      \item $\forall \Diamond \phi \in \Gamma \:\: \forall x \in W' \:\: \mathcal{M}, x \models \Diamond \phi \Rightarrow \exists y \in R'(x) \:\: \mathcal{M}, y \models \phi$
    \end{itemize}
  \end{defin}

Для общей эрудиции: селективная фильтрация является релятивизацией того, что в теории моделей называется критерием Тарского-Вота для элементарных подмоделей.
\end{frame}

\begin{frame}
    \frametitle{Лемма о селективной фильтрации}

    \begin{lem}
      Пусть $\mathcal{M} = \langle W, R, \vartheta \rangle$, $\Gamma$ --- множество формул, замкнутое относительно подформул и
      $\mathcal{M}' = \langle W', R', \vartheta' \rangle$ --- это селективная фильтрация модели $\mathcal{M}$ по $\Gamma$.

      \begin{itemize}
        \item $\forall \phi \in \Gamma \:\: \forall x \in W' \:\: \mathcal{M}, x \models \phi \Leftrightarrow \mathcal{M}', x \models \phi$
      \end{itemize}
    \end{lem}

Рассмотрим случай $\phi \eqcirc \Diamond \psi$
    \begin{proof}
      \begin{enumerate}
      \item $(\Rightarrow)$ Пусть $\mathcal{M}, x \models \Diamond \psi$.
      Тогда найдется $y \in R'(x)$, что $\mathcal{M}, y \models \psi$. $\psi \in \Gamma$.
      Тогда
      по предположению индукции $\mathcal{M}', y \models \psi$, откуда $\mathcal{M}', x \models \Diamond \psi$.
      \item $(\Leftarrow)$ Пусть $\mathcal{M}', x \models \Diamond \phi$, тогда найдется $y \in R'(x)$, что $\mathcal{M}', y \models \psi$. Тогда легко видеть, что $\mathcal{M}, x \models \Diamond \phi$.
    \end{enumerate}
  \end{proof}
\end{frame}

\begin{frame}
  \frametitle{Полнота {\bf GL} по Крипке}

  \begin{theor}
    ${\bf GL} = Log(\mathcal{F})$, где $\mathcal{F}$ --- это класс транзитивных и н\"{е}тереовых шкал.
  \end{theor}

  \begin{proof}
    Пусть $\phi$, такая формула, что множество $\{ \phi \}$ ${\bf GL}$-непротиворечиво.
    Тогда найдется точка $\Gamma \in W_{\bf GL}$, что $\phi \in \Gamma$.

    Тогда $\Diamond \phi \not\in \Gamma$ или $\Diamond (\phi \land \neg \Diamond \phi) \in \Gamma$.
    Пусть $\Delta \in W_{\bf GL}$, такое, что $\phi \in \Delta$ и $\neg \Diamond \phi \in \Delta$.

    Пусть $V_{\phi} = \{ \Theta \: | \: \mathcal{M}_{\bf GL}, \Theta \models \psi, \text{ для некоторого }\psi \in Sub(\phi) \}$.
  \end{proof}
\end{frame}

\begin{frame}
  \frametitle{Полнота {\bf GL} по Крипке}

  \begin{lem} $\mathcal{M}_{\bf GL} \upharpoonright V_{\phi}$ --- это селективная фильтрация $\mathcal{M}_{\bf GL}$ по $Sub(\phi)$
  \end{lem}

  \begin{proof}
    Пусть $\Diamond \psi$ --- это подформула $\phi$, $\Theta \in V_{\phi}$ и $\mathcal{M}_{\bf GL}, \Theta \models \Diamond \psi$. Тогда
    $\mathcal{M}_{\bf GL}, \Theta \models \Diamond (\psi \land \neg \Diamond \psi)$.
    Тогда найдется $\Xi \in V_{\phi}$, что $\Theta R_{\bf GL} \Xi$ и $\mathcal{M}_{\bf GL}, \Xi \models \psi$.
    Но при этом $\Xi \in (R_{\bf GL} \upharpoonright V_{\phi}) (\Theta)$.
  \end{proof}
\end{frame}

\begin{frame}
  \frametitle{Полнота {\bf GL} по Крипке}
  \begin{lem}
    $\mathcal{F}_{\bf GL} \upharpoonright V_{\phi} \models \Diamond p \to \Diamond (p \land \neg \Diamond p)$
  \end{lem}

  \begin{proof}
    Заметим, что $\mathcal{F}_{\bf GL} \upharpoonright V_{\phi}$ иррефлексивна по построению.
    Также заметим, что н\"{е}теровость отношения следует из того факта, что длина каждой возрастающей цепи мажорируется числом подформул $\phi$.
  \end{proof}
\end{frame}

  \begin{frame}
    \frametitle{Полнота ${\bf GL}$}
    \begin{theor}
      ${\bf GL} = Log(\mathbb{F})$, где $\mathbb{F}$ --- это класс всех транзитивных и н\"{е}теровых шкал.
    \end{theor}

  \begin{proof}
    Следует из лемм выше.
  \end{proof}
  \end{frame}

\begin{frame}
  \frametitle{На следующей лекции мы}
  \begin{itemize}
    \item Докажем теорему Харропа, которая связывает разрешимость с конечной аксиоматизируемостью и полнотой в конечных шкалах
    \item Рассмотрим разновидности фильтраций, специальных процедур, которые позволяют преобразовывать модели в конечные
    \item Рассмотрим такие конструкции, как порожденные подшкалы/подмодели, развертка и конус в шкале
    \item Покажем разрешимость логик ${\bf K}$, ${\bf T}$, ${\bf K}4$, ${\bf S}4$, ${\bf S}4.2$, ${\bf S}5$ и ${\bf GL}$
  \end{itemize}
\end{frame}

\end{document}
