\documentclass[pdf,utf8,russian,aspectratio=169]{beamer}
\usepackage[T2A]{fontenc}
\usetheme{Copenhagen}
\usepackage{setspace}
\usepackage{amsmath}
\usepackage{pgfplots}
\usepackage[utf8]{inputenc}
\usepackage{tikz-cd}
\usepackage[all, 2cell]{xy}
\usepackage{amssymb}
\usepackage{verbatim}
\usepackage[all]{xy}
\usepackage{tikz}
\usepackage{bussproofs}
\usepackage{dsfont}
\usepackage{mathabx}
\usepackage{animate}
\usetikzlibrary{graphs}
\usetikzlibrary{arrows}
\usepackage{hyperref}
\usepackage[english,russian]{babel}
\usepackage{listings}
\usepackage{color}
\usepackage[all, 2cell]{xy}
\usepackage[all]{xy}
\usepackage{listings}
\usepackage{mathrsfs}
\newtheorem{defin}{Определение}
\newtheorem{theor}{Теорема}
\newtheorem{lem}{Лемма}
\newtheorem{prop}{Предложение}
\newtheorem{col}{Следствие}
\title{Введение в модальную логику, \\ Лекция 3}
\author{Даня Рогозин \\ МГУ, Serokell}
\date{Computer Science Club}
\begin{document}

\maketitle

\begin{frame}
  \frametitle{На прошлой лекции мы}

  \begin{itemize}
    \item Ввели понятие нормальной модальной логики и сформулировали минимальную нормальную модальную логику, логику ${\bf K}$
    \item Ввели каноническую модель, доказали лемму Линденбаума и лемму о канонической модели
    \item Показали, что ${\bf K}$ является логикой классой шкал
  \end{itemize}
\end{frame}

\begin{frame}
  \frametitle{Модальные формулы и свойства шкал}

  Можно исследовать взаимосвязи общезначимости тех или иных формул со свойствами шкал, на которых данные формулы общезначимы. Введем следующие формулы:

  \begin{itemize}
    \item {\bf AT} $\Box p \to p$ (рефлексивность)
    \item {\bf A4} $\Box p \to \Box \Box p$ (транзитивность)
    \item {\bf AB} $p \to \Box \Diamond p$ (симметричность)
    \item {\bf ACR} $\Diamond \Box p \to \Box \Diamond p$ (формула Черча-Россера)
    \item {\bf AD} $\Diamond \top$ (сериальность)
  \end{itemize}
\end{frame}

\begin{frame}
    \frametitle{Модальные формулы и свойства шкал}

    \begin{lem} Пусть $\mathcal{F} = \langle W, R \rangle$ --- это шкала Крипке, тогда
      \begin{enumerate}
        \item $\mathcal{F} \models \Box p \to p \Leftrightarrow \forall x \in W \:\: x R x$
        \item $\mathcal{F} \models \Box p \to \Box \Box p \Leftrightarrow \forall x, y, z \in W \:\: x R y \: \& \: y R z \Rightarrow x R z$
        \item $\mathcal{F} \models p \to \Box \Diamond p \Leftrightarrow \forall x, y \in W \:\: x R y \Rightarrow y R x$
        \item $\mathcal{F} \models \Diamond \Box p \to \Box \Diamond p \Leftrightarrow
        \forall x, y, z \in W \:\: x R y \: \& \: x R z \Rightarrow \exists z_1 \in W \: y R z_1 \: \& \: z R z_1$
        \item $\mathcal{F} \models \Diamond \top \Leftrightarrow \forall x \in W \: \exists y \in W \:\: x R y$
      \end{enumerate}
    \end{lem}

    \begin{proof}
      Рассмотрим эквивалентности (1) и (2).
    \end{proof}
\end{frame}

\begin{frame}
    \frametitle{Модальные формулы и свойства шкал}
    \begin{lem} Пусть $\mathcal{F} = \langle W, R \rangle$ --- это шкала Крипке, тогда

      $\mathcal{F} \models \Box p \to p \Leftrightarrow \forall x \in W \:\: x R x$
    \end{lem}

    Для удобства перепишем эту формулу в эквивалентный вид $p \to \Diamond p$.

    \begin{proof}
      $(\Rightarrow)$ Пусть $\mathcal{F} \models p \to \Diamond p$. Рассмотрим оценку $\vartheta(p) = \{ x \}$, где $x \in W$.
      Тогда $\mathcal{M}, x \models p$ и $\mathcal{M}, x \models \Diamond p$. Тогда найдется $y \in R(x)$, такой, что $\mathcal{M}, y \models p$.
      Но $p$ истинно только в $x$, тогда $x R x$
    \end{proof}
\end{frame}

\begin{frame}
    \frametitle{Модальные формулы и свойства шкал}
    \begin{lem} Пусть $\mathcal{F} = \langle W, R \rangle$ --- это шкала Крипке, тогда

      $\mathcal{F} \models \Box p \to p \Leftrightarrow \forall x \in W \:\: x R x$
    \end{lem}

    \begin{proof}
      $(\Leftarrow)$ Пусть $\mathcal{F} = \langle W, R \rangle$ рефлексивная шкала и $\mathcal{F} \not\models p \to \Diamond p$.
      Тогда найдется такая оценка $\vartheta$, модель $\mathcal{M} = \langle \mathcal{F}, \vartheta \rangle$ и $x \in W$, что
      $\mathcal{M}, x \models p$ и $\mathcal{M}, x \not\models \Diamond p$. Из $\mathcal{M}, x \not\models \Diamond p$ следует, что для любого
      $y \in R(x)$, $\mathcal{M}, y \not\models p$. Так как $\mathcal{F}$ рефлексивная шкала, тогда $x R x$ и $x \in R(x)$, тогда $\mathcal{M}, x \not\models p$. Противоречие.
    \end{proof}
\end{frame}

\begin{frame}
  \frametitle{Модальные формулы и свойства шкал}

  \begin{lem} Пусть $\mathcal{F} = \langle W, R \rangle$ --- это шкала Крипке, тогда

    $\mathcal{F} \models \Box p \to \Box \Box p \Leftrightarrow \forall x, y, z \in W \:\: x R y \: \& \: y R z \Rightarrow x R z$
  \end{lem}

  Аналогично, перепишем формулу как $\Diamond \Diamond p \to \Diamond p$

\begin{proof}
  ($\Rightarrow$) Пусть $\mathcal{F} \models \Diamond \Diamond p \to \Diamond p$ и $x, y, z \in W$, такие, что $x R y$ и $y R z$. Рассмотрим оценку
  $\vartheta(p) = \{ z \}$. Так как $x R y R z$, тогда $\mathcal{M}, x \models \Diamond \Diamond p$. Тогда $\mathcal{M}, x \models \Diamond p$.
  Значит, найдется $x' \in R(x)$, что $\mathcal{M}, x' \models p$. Но $p$ истинно только $z$, тогда $x R z$
\end{proof}
\end{frame}

\begin{frame}
  \frametitle{Модальные формулы и свойства шкал}

  \begin{lem} Пусть $\mathcal{F} = \langle W, R \rangle$ --- это шкала Крипке, тогда

    $\mathcal{F} \models \Box p \to \Box \Box p \Leftrightarrow \forall x, y, z \in W \:\: x R y \: \& \: y R z \Rightarrow x R z$
  \end{lem}

\begin{proof} Пусть $\mathcal{F} = \langle W, R \rangle$ --- это транзитивная шкала,
  $\vartheta$ --- это оценка, модель $\mathcal{M} = \langle \mathcal{F}, \vartheta \rangle$ и $x \in W$. Пусть $\mathcal{M}, x \models \Diamond \Diamond p$. Тогда
  найдется $y \in R(x)$, что $\mathcal{M}, y \models \Diamond p$, откуда $\mathcal{M}, z \models p$ для некоторого $z \in R(x)$. $x R y$ и $y R z$, тогда $x R z$,
  значит $\mathcal{M}, x \models \Diamond p$.
\end{proof}
\end{frame}

\begin{frame}
  \frametitle{Модальные логики}

  Вернемся к списку формул:

  \begin{itemize}
    \item {\bf AT} $\Box p \to p$ (рефлексивность)
    \item {\bf A4} $\Box p \to \Box \Box p$ (транзитивность)
    \item {\bf AB} $p \to \Box \Diamond p$ (рефлексивность)
    \item {\bf ACR} $\Diamond \Box p \to \Box \Diamond p$ (формула Черча-Россера)
    \item {\bf AD} $\Diamond \top$ (сериальность)
  \end{itemize}

  Определим список логик:
\end{frame}

\begin{frame}
  \frametitle{Модальные логики}

  Примеры модальных логик:

\begin{itemize}
  \item ${\bf T} = {\bf K} \oplus \Box p \to p$
  \item ${\bf K}4 = {\bf K} \oplus \Box p \to \Box \Box p$
  \item ${\bf D} = {\bf K} \oplus \Diamond \top$
  \item ${\bf S}4 = {\bf K}4 \oplus \Box p \to p = {\bf T} \oplus \Box p \to \Box \Box p$
  \item ${\bf S}5 = {\bf S}4 \oplus p \to \Box \Diamond p$
  \item ${\bf S}4.2 = {\bf S}4 \oplus \Diamond \Box p \to \Box \Diamond p$
\end{itemize}
\end{frame}

\begin{frame}
  \frametitle{Каноническая логика}

  \begin{defin}
    \begin{itemize}
    \item Формула $\phi$ называется канонической, если логика $\mathcal{L} = {\bf K} \oplus \phi = Log(\mathcal{F}_{\mathcal{L}})$
    \item Логика называется канонической, если $\mathcal{L} = Log(\mathcal{F}_{\mathcal{L}})$.
  \end{itemize}
  \end{defin}

  \begin{prop}
    Если логика каноническая, то она полна.
  \end{prop}
\end{frame}


\begin{frame}
  \frametitle{Лемма о каноничности формул}
  \begin{lem}
    Следующие формулы канонические:
    \begin{enumerate}
      \item {\bf AT} $\Box p \to p$
      \item {\bf A4} $\Box p \to \Box \Box p$
      \item {\bf AB} $p \to \Box \Diamond p$
      \item {\bf ACR} $\Diamond \Box p \to \Box \Diamond p$
      \item {\bf AD} $\Diamond \top$
    \end{enumerate}
  \end{lem}

  Рассмотрим случаи ${\bf A}4$, ${\bf AT}$ и ${\bf ACR}$.
\end{frame}

\begin{frame}
  \frametitle{Лемма о каноничности формул}

  \begin{lem}
    Формула $\Box p \to \Box \Box p$ канонична, то есть $\mathcal{F}_{{\bf K} 4} \models \Box p \to \Box \Box p$.
  \end{lem}

  Снова перепишем формулу ${\bf A}4$, а отношение $R_{{\bf K} 4}$ в канонической шкале эквивалентным образом переопределим как $\Gamma R_{{\bf K}4} \Delta \Leftrightarrow \phi \in \Delta \Rightarrow \Diamond \phi \in \Gamma$

  \begin{proof}
    Пусть $\Gamma, \Delta, \Theta \in W_{{\bf K}4}$, такие что $\Gamma R_{{\bf K} 4} \Delta$ и $\Delta R_{{\bf K} 4} \Theta$. Покажем, что если $\phi \in \Theta$,
    то $\Diamond \phi \in \Gamma$. Пусть $\phi \in \Theta$, тогда, по лемме о канонической модели, $\mathcal{M}_{{\bf K}4}, \Theta \models \phi$. Тогда
    $\mathcal{M}_{{\bf K}4}, \Delta \models \Diamond \phi$, откуда $\mathcal{M}_{{\bf K}4}, \Gamma \models \Diamond \Diamond \phi$. С другой стороны,
    $\mathcal{M}_{{\bf K}4}, \Gamma \models \Diamond \Diamond \phi \to \Diamond \phi$ по лемме о канонической модели.
    Тогда $\mathcal{M}_{{\bf K}4}, \Gamma \models \Diamond \phi$, тогда $\Diamond \phi \in \Gamma$.
  \end{proof}
\end{frame}

\begin{frame}
  \frametitle{Лемма о каноничности формул}

  \begin{lem}
    Формула $\Box p \to p$ канонична, то есть $\mathcal{F}_{T} \models \Box p \to p$.
  \end{lem}

\begin{proof}
  Пусть $\Gamma \in W_{\bf T}$ и $\mathcal{M}_{\bf T}, \Gamma \models p$.
  $\mathcal{M}_{\bf T}, \Gamma \models p \to \Diamond p$ по лемме о канонической модели и
  $\mathcal{M}_{\bf T}, \Gamma \models \Diamond p$, тогда $\Diamond p \in \Gamma$, значит, $\Gamma R_{\bf T} \Gamma$.
\end{proof}

\end{frame}

\begin{frame}
  \frametitle{Лемма о каноничности формул}
  \begin{lem}
    Пусть $\mathcal{L} = {\bf K} + {\bf ACR}$, $\mathcal{F}_{L} \models {\bf ACR}$, где ${\bf ACR} = \Diamond \Box p \to \Box \Diamond p$
  \end{lem}

  \begin{proof}
    Пусть $\Gamma, \Delta, \Theta \in W_{\mathcal{L}}$, такие, что $\Gamma R_{\mathcal{L}} \Delta$ и $\Gamma R_{\mathcal{L}} \Theta$.
    Тогда $\Diamond \Delta \subseteq \Gamma$ и $\Diamond \Theta \subseteq \Gamma$. Найдем $\Xi$, такой, что
    $\Box \Delta \subseteq \Xi$ и $\Box \Theta \subseteq \Xi$. Положим $\Xi = \Box \Delta \cup \Box \Theta$ и покажем, что
    $\Xi$ $\mathcal{L}$-непротиворечиво.

    Предположим противное, тогда найдутся такие формулы $\phi \in \Box \Delta$ и $\psi \in \Box \Theta$, что $\mathcal{L} \vdash \phi \to \neg \psi$.
    Так как $\phi \in \Box \Delta$, тогда $\Box \phi \in \Delta$. С другой стороны, $\Diamond \Delta \subseteq \Gamma$, тогда $\Diamond \Box \phi \in \Gamma$.

    $\Diamond \Box \phi \to \Box \Diamond \phi \in \Gamma$, тогда $\Box \Diamond \phi \in \Gamma$. По монотонности $\Box$ и $\Diamond$,
    $\mathcal{L} \vdash \Box \Diamond \phi \to \Box \Diamond \neg \psi$. Тогда $\Box \Diamond \neg \psi \in \Gamma$, откуда $\Diamond \neg \psi \in \Theta$ ввиду
    $\Gamma R_{\mathcal{L}} \Theta$, следовательно, $\neg \psi \in \Gamma$. С другой стороны, $\psi \in \Box \Theta$, тогда $\psi \in \Gamma$. Противоречие.
  \end{proof}
\end{frame}

\begin{frame}
  \frametitle{Лемма о каноничности логик}

Вернемся к списку логик:

  \begin{itemize}
    \item ${\bf T} = {\bf K} \oplus \Box p \to p$
    \item ${\bf K}4 = {\bf K} \oplus \Box p \to \Box \Box p$
    \item ${\bf D} = {\bf K} \oplus \Diamond \top$
    \item ${\bf S}4 = {\bf K}4 \oplus \Box p \to p = {\bf T} \oplus \Box p \to \Box \Box p$
    \item ${\bf S}5 = {\bf S}4 \oplus p \to \Box \Diamond p$
    \item ${\bf S}4.2 = {\bf S}4 \oplus \Diamond \Box p \to \Box \Diamond p$
  \end{itemize}

Следующее утверждение легко следует из леммы о каноничности формул:

\begin{lem}
  Логики из списка выше являются каноническими
\end{lem}

\end{frame}

\begin{frame}
  \frametitle{Теорема о полноте}

  Имеет место следующая теорема:

\begin{theor}
  \begin{enumerate}
    \item ${\bf T} = Log(\mathbb{F})$, где $\mathbb{F}$ --- класс всех рефлексивных шкал.
    \item ${\bf K}4 = Log(\mathbb{F})$, где $\mathbb{F}$ --- класс всех транзитивных шкал.
    \item ${\bf D} = Log(\mathbb{F})$, где $\mathbb{F}$ --- класс всех сериальных шкал.
    \item ${\bf S}4 = Log(\mathbb{F})$, где $\mathbb{F}$ --- класс всех предпорядков.
    \item ${\bf S}5 = Log(\mathbb{F})$, где $\mathbb{F}$ --- класс всех отношений эквивалентности.
    \item ${\bf S}4.2 = Log(\mathbb{F})$, где $\mathbb{F}$ --- класс всех конфлюентных предпорядков.
  \end{enumerate}
\end{theor}

\begin{proof}
  Рассмотрим в качестве примера логику ${\bf S}4$
\end{proof}
\end{frame}

\begin{frame}
  \frametitle{Теорема о полноте}
  \begin{theor}
    ${\bf S}4 = Log(\mathbb{F})$, где $\mathbb{F}$ --- класс всех предпорядков.
  \end{theor}
  \begin{proof}
    Рассмотрим следующие включения:

    ${\bf S}4 \subseteq Log(\mathbb{F}) \subseteq Log(\mathcal{F}_{{\bf S}4}) \subseteq {\bf S}4$

    Первое включение --- это теорема корректности для ${\bf S}4$, третье включение (и, более того, равенство) следует из того факта, что ${\bf S}4$ является канонической логикой. Второе включение следует из того факта, что отношение $R_{{\bf S}4}$ является предпорядком в канонической шкале.
  \end{proof}
\end{frame}

\begin{frame}
  \frametitle{Пример неканонической логики}

  Мы сказали, что всякая каноническая логика полна. Рассмотрим пример логики, которая полна по Крипке, но является канонической.
\end{frame}

\begin{frame}
  \frametitle{Н\"{e}терово отношение}
\begin{defin}
  Отношение $R \subseteq W \times W$ является обратно фундированным (н\"{e}теровым), если не существует бесконечных цепей $a_1 R a_2 R \dots$.
\end{defin}
\end{frame}

\begin{frame}
  \frametitle{Эквивалентные определения н\"{e}теровости}
\begin{lem}
  Отношение $R \subseteq W \times W$ является н\"{е}теровым $\Leftrightarrow$ каждое непустое подмножество $W$ имеет $R$-максимальный элемент.
\end{lem}

\begin{proof}
  ($\Leftarrow$) Пусть отношение $R$ не является н\"{е}теровым, тогда найдется бесконечная цепь $a_1 R a_2 R \dots$. Тогда множество $W' = \{ a_1, a_2, \dots \}$
не будет содержать максимального элемента.
\end{proof}
\end{frame}

\begin{frame}
  \frametitle{Эквивалентные определения н\"{e}теровости}
\begin{lem}
  Отношение $R \subseteq W \times W$ является н\"{е}теровым $\Leftrightarrow$ каждое непустое подмножество $W$ имеет $R$-максимальный элемент.
\end{lem}

WARNING: обратная импликация требует аксиомы выбора.

\begin{proof}
  ($\Rightarrow$) Пусть $W'$ не имеет максимального элемента. Пусть $a_0 \in W'$, тогда найдется $a_1$, такой, что $a_0 R a_1$. Дальше, для любого $n \in \mathbb{N}$, $a_n R a_{n + 1}$. По аксиоме выбора, можно построить последовательность $\{ a_n \}_{n \in \mathbb{N}}$, которая по отношению $R$ и образует бесконечную последовательность $a_0 R a_1 R a_2 \dots$.
\end{proof}
\end{frame}

\begin{frame}
  \frametitle{Формула Л\"{е}ба}

  \begin{defin}
    Формулой Л\"{е}ба называется формула вида $\Box (\Box p \to p) \to \Box p$, или, что эквивалентно, $\Diamond p \to \Diamond (p \land \neg \Diamond p)$.
  \end{defin}

\begin{defin} Логика Г\"{е}деля-Л\"{е}ба
  ${\bf GL} = {\bf K} + \Box (\Box p \to p) \to \Box p$
\end{defin}

Логика Г\"{е}деля-Л\"{е}ба применяется в теории доказательств, если $\Box$ понимать как доказуемость в г\"{е}делевых теориях. Смотрите:

\end{frame}

\begin{frame}
  \frametitle{Н\"{e}теровость шкал}
\begin{lem}
  Пусть $\mathcal{F} = \langle W, R \rangle$ --- это шкала Крипке, тогда

\begin{itemize}
  \item $\mathcal{F} \models \Box (\Box p \to p) \to \Box p \Leftrightarrow R \text{ транзитивно и н\"{е}терово}$
\end{itemize}
\end{lem}

\begin{proof}
  $(\Rightarrow)$. Покажем транзитивность.

  Пусть $x, y, z \in W$ и $x R y R z$. Положим $\vartheta(p) = W \setminus \{ y, z \}$ и $\mathcal{M} = \langle \mathcal{F}, \vartheta \rangle$. Тогда
  $\mathcal{M}, x \not\models \Box p$ и $\mathcal{M}, y \not\models p$. При этом, ясно, что $\mathcal{M}, x \models \Box (\Box p \to p) \to \Box p$.

  Тогда $\mathcal{M}, x \not\models \Box (\Box p \to p)$, тогда найдется $z' \in R(x)$, что $\mathcal{M}, z' \models \Box p$ и $\mathcal{M}, z' \not\models p$.

  Значит, $z' = y$ или $z' = z$. Первое равенство неверно, так как $\mathcal{M}, y \not\models p$. Остается только $z$, тогда $x R z$.
\end{proof}
\end{frame}

\begin{frame}
  \frametitle{Н\"{e}теровость шкал}
\begin{lem}
  Пусть $\mathcal{F} = \langle W, R \rangle$ --- это шкала Крипке, тогда

\begin{itemize}
  \item $\mathcal{F} \models \Box (\Box p \to p) \to \Box p \Leftrightarrow R \text{ транзитивно и н\"{е}терово}$
\end{itemize}
\end{lem}

\begin{proof}
  $(\Rightarrow)$. Покажем н\"{e}теровость.

  Пусть $W' \subseteq W$ непустое подмножество и $x \in W'$. Положим $\vartheta(p) = W \setminus W'$. Рассмотрим два случая:

\begin{enumerate}
  \item Пусть $\mathcal{M}, x \not\models \Box p$, тогда $\mathcal{M}, x \not\models \Box (\Box p \to p)$. Значит найдется $y \in R(x)$ такой,
  что $\mathcal{M}, y \models \Box p$ и $\mathcal{M}, y \not\models p$. Тогда $y$ максимальный элемент $W'$.
  \item Пусть $\mathcal{M}, x \models \Box p$, но $\mathcal{M}, x \not\models p$, то $x$ максимальный элемент $W'$.
\end{enumerate}
\end{proof}
\end{frame}

\begin{frame}
  \frametitle{Н\"{e}теровость шкал}

  \begin{lem}
    Пусть $\mathcal{F} = \langle W, R \rangle$ --- это шкала Крипке, тогда

  \begin{itemize}
    \item $\mathcal{F} \models \Box (\Box p \to p) \to \Box p \Leftrightarrow R \text{ транзитивно и н\"{е}терово}$
  \end{itemize}
\end{lem}

\begin{proof}
  $(\Leftarrow)$ Пусть $\mathcal{F}$ транзитивна и н\"{e}терова и $\vartheta$ --- это оценка. Пусть $x \in W$ и $\mathcal{M}, x \models \Diamond p$.
  Положим $W' = \vartheta(p) \cap R(x)$, которое непусто. Пусть $y$ --- максимальный элемент $W'$, тогда $\mathcal{M}, y \models p$ и
  $\mathcal{M}, y \not\models \Diamond p$, то есть $\mathcal{M}, y \models \neg \Diamond p$, тогда $\mathcal{M}, y \models p \land \neg \Diamond p$.
  Откуда $\mathcal{M}, x \models \Diamond p \to \Diamond (p \land \neg \Diamond p)$.
\end{proof}
\end{frame}

\begin{frame}
  \frametitle{Н\"{e}теровость шкал}
  \begin{col}
    \begin{enumerate}
    \item Если $\mathcal{F} \models \Box (\Box p \to p) \to \Box p$, тогда $\mathcal{F}$ иррефлексивна, то есть $\forall x \in W \:\: \neg (x R x)$
    \item Если $\mathcal{F} \models \Box (\Box p \to p) \to \Box p$, тогда $\mathcal{F} \not\models \Box p \to p$
  \end{enumerate}
  \end{col}

  \begin{proof}
    \begin{enumerate}
    \item Пусть $\mathcal{F} \models \Box (\Box p \to p) \to \Box p$ и существует $x \in W$, такой что $x R x$. Тогда данная шкала является н\"{e}теровой.
    С другой стороны, существует бесконечно возрастающая цепь $x R x R x \dots$, противоречие.
    \item Очевидно.
  \end{enumerate}
  \end{proof}
\end{frame}

\begin{frame}
  \frametitle{Неканоничность формулы Л\"{е}ба}

  \begin{lem}
    $\mathcal{F}_{\bf GL} \not\models \Box (\Box p \to p) \to \Box p$
  \end{lem}

  \begin{proof}
    Положим $\Gamma = \{ \Box \phi \to \phi \: | \: \phi \in Fm \}$ и покажем, что $\Gamma$ ${\bf GL}$-непротиворечиво.
    Предположим, что $\Gamma$ противоречиво, тогда найдутся формулы $\phi_1, \dots, \phi_n$, что ${\bf GL} \vdash \neg \land_{i = 1}^n (\Box \phi_i \to \phi_i)$,
    откуда ${\bf GL} \vdash \lor_{i=1}^{n} (\Box \phi_i \land \neg \phi_i)$. Пусть $\mathcal{F} = \langle n + 1, < \rangle$, где $n + 1 = \{ 0, \dots, n \}$.
    Тогда $\mathcal{F} \models \lor_{i=1}^{n} (\Box \phi_i \land \neg \phi_i)$. Эта формула состоит из $n$ дизъюнктов, тогда найдется $i \in \{ 1, \dots, n \}$,
    что $\Box \phi_i \land \neg \phi_i$ истинно при некоторых $x, y \in n + 1$, что $x < y$. Тогда в $x$ верны $\Box \phi_i$ и $\Diamond \neg \phi_i$. Противоречие.
  \end{proof}
\end{frame}

\begin{frame}
  \frametitle{Селективная фильтрация}
  \begin{defin}
    Пусть $\mathcal{M} = \langle W, R, \vartheta \rangle$ и $\Gamma$ --- множество формул, замкнутое относительно подформул.
    Селективной фильтрацией модели $\mathcal{M}$ по множеству формул $\Gamma$ называется модель $\mathcal{M} = \langle W', R', \vartheta' \rangle$, где
    $W' \subseteq W$, $R' \subseteq R$ и $\vartheta'(p) = \vartheta(p) \cap W'$ для $p \in \Gamma$ со следующим условием:

    \begin{itemize}
      \item $\forall \Diamond \phi \in \Gamma \:\: \forall x \in W' \:\: \mathcal{M}, x \models \Diamond \phi \Rightarrow \exists y \in R'(x) \:\: \mathcal{M}, y \models \phi$
    \end{itemize}
  \end{defin}
\end{frame}

\begin{frame}
    \frametitle{Лемма о селективной фильтрации}

    \begin{lem}
      Пусть $\mathcal{M} = \langle W, R, \vartheta \rangle$, $\Gamma$ --- множество формул, замкнутое относительно подформул и
      $\mathcal{M}' = \langle W', R', \vartheta' \rangle$ --- это селективная фильтрация модели $\mathcal{M}$ по $\Gamma$. Тогда:

      \begin{itemize}
        \item $\forall \phi \in \Gamma \:\: \forall x \in W' \:\: \mathcal{M}, x \models \phi \Leftrightarrow \mathcal{M}', x \models \phi$
      \end{itemize}
    \end{lem}
Рассмотрим случай $\phi \eqcirc \Diamond \psi$
    \begin{proof}
      \begin{enumerate}
      \item $(\Rightarrow)$ Пусть $\mathcal{M}, x \models \Diamond \psi$, тогда найдется $y \in R'(x)$, что $\mathcal{M}, y \models \psi$. $\psi \in \Gamma$, тогда
      по предположению индукции $\mathcal{M}', y \models \psi$, откуда $\mathcal{M}', x \models \Diamond \psi$.
      \item $(\Leftarrow)$ Пусть $\mathcal{M}', x \models \Diamond \phi$, тогда найдется $y \in R'(x)$, что $\mathcal{M}', y \models \psi$. Тогда легко видеть, что
      $\mathcal{M}, x \models \Diamond \phi$.
    \end{enumerate}
    \end{proof}
\end{frame}

\begin{frame}
  \frametitle{Полнота {\bf GL} по Крипке}
\end{frame}

\end{document}
